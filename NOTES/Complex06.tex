\input{PREAMBLE}

% -------------------------
\begin{document}

\title{%
{\small TF2201 Matematika Rekayasa II}\\
Variabel Kompleks: Residu
}
\author{Fadjar Fathurrahman\\
Teknik Fisika\\
Institut Teknologi Bandung}
\date{2023}
\maketitle

Dalam bagian ini kita akan mempelajari suatu teknik yang berguna untuk
menghitung beberapa jenis integral dari variabel real.

Misalkan $f(z)$ singular pada titik $z=a$ dan analitik pada titik
yang lain di dalam suatu lingkaran yang berpusat pada $z=a$. Maka
$f(z)$ dapat diekspansi ke dalam deret Laurent:
\begin{equation*}
f(z) = \cdots + \frac{b_{m}}{(z-a)^{m}} + \cdots + \frac{b_{2}}{(z-a)^{2}} +
\frac{b_{1}}{z-a}+a_{0}+a_{1}(z-a)+a_{2}(z-a)^{2}+\cdots
\end{equation*}

Ada tiga kasus yang dapat muncul.

Pertama, semua koefisien $b_{1},b_{2},\cdots$ bernilai nol. Maka
$f(z)$ disebut memiliki singularitas yang dapat dibuang (\textit{removable singularity}).
Contohnya: fungsi $\sin(z)/z$ memiliki singularitas yang dapat
dibuang pada $z=0$.

Kedua, hanya ada beberapa (terbatas) \textbf{$b_{n}$} bernilai tidak
nol. Dengan kata lain $f(z)$ memiliki pole pada $z=a$. Jika $f(z)$ memiliki
suatu pole, maka
\begin{equation*}
f(z)=\frac{b_{m}}{(z-a)^{m}}+\cdots+\frac{b_{1}}{z-a}+a_{0}+a_{1}(z-a)+\cdots
\end{equation*}
dengan $b_{m}\neq0$.
Dalam kasus ini kita menyebut bahwa pole pada $z=a$ memiliki orde-$m$.

Ketiga, terdapat $b_{m}$ sejumlah tak-hingga yang tidak nol, sehingga
dikatakan bahwa $f(z)$ memiliki singularitas esensial pada $z=a$.
Fungsi $e^{1/z}$ memiliki ekspansi deret Laurent:
\begin{equation*}
e^{1/z} = 1 + \frac{1}{z} + \frac{1}{2!z^{2}} + \cdots + \frac{1}{n!z^{n}}+\cdots
\end{equation*}
yang valid untuk semua $z$, $|z|>0$. Titik $z = 0$ adalah singularitas
esensial dari $e^{1/z}$.

Fungsi rasional hanya memiliki pole atau singularitas yang dapat dibuang
saja sebagai jenis singularitasnya.

Koefisien $b_{1}$ dapat dihitung dari:
\begin{equation*}
b_{1} = \frac{1}{2\pi\imath}\oint_{C_{1}}f(w)\ \mathrm{d}w
\end{equation*}
Jadi integral dari suatu fungsi $f(z)$ pada suatu kurva $C_{1}$
yang meliputi suatu titik singular diberikan oleh
\begin{equation*}
\oint f(z)\ \mathrm{d}z=2\pi\imath b_{1}
\label{eq:Potter_Eq_10_10_5}
\end{equation*}
di mana $b_{1}$ adalah koefisien dari suku $(z-a)^{-1}$ pada ekspansi
deret Laurent pada titik $z=a$. Kuantitas $b_{1}$ disebut sebagai
residu dari $f(z)$ pada $z=a$. Oleh karena itu, untuk menemukan
integral dari suatu fungsi di pada suatu kontur di sekitar titik singular,
kita hanya perlu menemukan ekspansi deret Laurent dan menggunakan
Persamaan \ref{eq:Potter_Eq_10_10_5}.
Jika ada lebih dari satu singularitas, maka kita dapat
membuat daerah tersebut menjadi terhubung sederhana seperti pada Gambar
\ref{eq:Potter_Ch10_Fig_18}.
%
\begin{figure}[h]
{\centering
\includegraphics{../images_priv/Potter_Ch10_Fig_18.png}
\par}
\caption{Integrasi pada suatu kurva yang meliputi beberapa titik singular.}
\label{eq:Potter_Ch10_Fig_18}
\end{figure}
%
Kemudian dengan menggunakan Teorema Integral Cauchy diperoleh
\begin{equation*}
\oint_{C}f(z)\ \mathrm{d}z+\oint_{C_{1}}f(z)\ \mathrm{d}z+\oint_{C_{2}}f(z)\ \mathrm{d}z +
\oint_{C_{3}}f(z)\ \mathrm{d}z=0
\end{equation*}
karena $f(z)$ analitik pada semua titik di luar lingkaran kecil dan
di dalam $C$. Jika jika membalikkan arah integrasi pada kontur di
sekitar lingkaran kecil diperoleh
\begin{equation*}
\oint_{C}f(z)\ \mathrm{d}z=\oint_{C_{1}}f(z)\ \mathrm{d}z + 
\oint_{C_{2}}f(z)\ \mathrm{d}z + \oint_{C_{3}}f(z)\ \mathrm{d}z
\end{equation*}
Dalam residu pada setiap titik, kita memperoleh Teorema Residu Cauchy:
\begin{equation*}
\oint_{C}f(z)\ \mathrm{d}z = 2\pi\imath\left[\left(b_{1}\right)_{a_{1}} + \left(b_{1}\right)_{a_{2}} +
\left(b_{1}\right)_{a_{3}}\right]
\end{equation*}
di mana $b_{1}$ adalah koefisien dari $(z-a)^{-1}$ pada ekspansi
deret Laurent pada masing-masing titik.

Teknik lain, yang sering digunakan untuk menemukan residur pada suatu
titik singular adalah dengan mengalikan deret Laurent dengan $(z-a)^{m}$
untuk mendapatkan
\begin{equation*}
(z-a)^{m}f(z) = b_{m} + b_{m-1}(z-a) + \cdots + b_{1}(z-a)^{m-1} +
a_{0}(z-a)^{m} + a_{1}(z-a)^{m+1} + \cdots
\end{equation*}
Jika deret ini didiferensiasi sebanyak $(m-1)$ kali dan substitusi
$z=a$ diperoleh residu
\begin{equation*}
b_{1}=\frac{1}{(m-1)!}\left\{ \frac{\mathrm{d}^{m-1}}{\mathrm{d}z^{m-1}}\left[(z-a)^{m}f(z)\right]\right\} _{z=a}
\end{equation*}
Agar teknik ini berhasil, orde dari pole harus diketahui sebelumnya.
Jika $m=1$ maka tidak ada operasi diferensiasi yang diperlukan sehingga
diperoleh dari
\[
\lim_{z\rightarrow a}(z-a)f(z)
\]

Teorema residu dapat digunakan untuk mengevaluasi integral variabel
real dengan bentuk tertentu.

Misalnya, tinjau integral
\begin{equation*}
I = \int_{0}^{2\pi}g(\cos(\theta),\sin(\theta))\ \mathrm{d}\theta
\end{equation*}
di mana $g(\cos(\theta),\sin(\theta))$ adalah fungsi rasional dari
$\cos(\theta)$ dan $\sin(\theta)$ yang tidak memiliki singularitas
pada interval $0\leq\theta\leq2\pi$. Dengan menggunakan substitusi
\[
e^{\imath\theta}=z
\]
diperoleh
\begin{align*}
\cos(\theta) & =\frac{1}{2}\left(e^{\imath\theta}+e^{-\imath\theta}\right)=\frac{1}{2}\left(z+\frac{1}{z}\right)\\
\sin(\theta) & =\frac{1}{2\imath}\left(e^{\imath\theta}-e^{-\imath\theta}\right)=\frac{1}{2\imath}\left(z-\frac{1}{z}\right)\\
\mathrm{d}\theta & =\frac{\mathrm{d}z}{\imath e^{\imath\theta}}=\frac{\mathrm{d}z}{\imath z}
\end{align*}
Integral sekarang menjadi
\[
I=\oint_{C}\frac{f(z)}{\imath z}\ \mathrm{d}z
\]
Teorema residu dapat diaplikasikan untuk bentuk integral di atas setelah
$f(z)$ diperoleh. Semua residu yang ada pada lingkaran satuan harus
diperhitungkan.

Jenis kedua dari integral yang dapat dihitung dengan menggunakan teorema
residu adalah integral
\[
I=\int_{-\infty}^{\infty}f(x)\ \mathrm{d}x
\]
di mana $f(x)$ adalah fungsi rasional
\[
f(x)=\frac{p(x)}{q(x)}
\]
di mana $q(x)$ tidak memiliki real zero dan memiliki derajat sedikitnya
dua lebih besar daripada $p(x)$. Tinjau integral terkait
\begin{equation*}
I_{1} = \oint_{C}f(z)\ \mathrm{d}z
\end{equation*}
di mana $C$ adalah kontur tertutup seperti pada
Gambar \ref{fig:Potter_Ch10_Fig_20}.
\begin{figure}[h]
{\centering
\includegraphics{../images_priv/Potter_Ch10_Fig_20.png}
\par}
\caption{Kontur}\ref{fig:Potter_Ch10_Fig_20}
\end{figure}
Jika $C_{1}$ adalah bagian setengah lingkaran dari kurva $C$, maka 
\[
I_{1} = \int_{C_{1}}f(z)\ \mathrm{d}z + \int_{-R}^{R}f(x)\ \mathrm{d}x = 2\pi\imath\sum_{n=1}^{N}\left(b_{1}\right)_{n}
\]
di mana teorema residu Cauchy telah digunakan. Dalam persamaan ini,
$N$ merupakan jumlah singularitas yang berada pada setengah bidang
atas yang dilingkupi oleh setengah lingkaran. Sekarang, akan ditunjukkan
bahwa
\[
\int_{C_{1}}f(z)\ \mathrm{d}z \rightarrow 0
\]
untuk $R\rightarrow\infty$. Dengan menggunakan Persamaan XXX dan
syarat bahwa $q(z)$ memiliki derajat setidaknya dua lebih besar daripada
$p(z)$, maka
\[
|f(z)|=\left|\frac{p(z)}{q(z)}\right|=\frac{|p(z)|}{|q(z)|}\sim\frac{1}{R^{2}}
\]
maka
\[
\left|\int_{C_{1}}f(x)\ \mathrm{d}z\right|\leq\left|f_{\mathrm{max}}\right|\pi R\sim\frac{1}{R}
\]
Ketika $R\rightarrow\infty$, diperoleh
\[
\int_{C_{1}}f(z)\ \mathrm{d}z\rightarrow0
\]
sehingga
\[
\int_{-\infty}^{\infty}f(x)\ \mathrm{d}x=2\pi\imath\sum_{n=1}^{N}\left(b_{1}\right)_{n}
\]
di mana $b_{1}$ melibatkan residu dari $f(z)$ pada seluruh singularitas
yang ada pada setengah bidang atas.

Jenis ketiga dari integral yang dapat dievaluasi menggunakan teorema
residu adalah
\[
I=\int_{-\infty}^{\infty}f(x)\sin(x)\ \mathrm{d}x
\]
dan
\[
I=\int_{-\infty}^{\infty}f(x)\cos(x)\ \mathrm{d}x
\]

Untuk menghitung jenis integral ini, tinjau integral kompleks
\[
I_{1} = \oint_{C} f(z) e^{\imath mz}\ \mathrm{d}z
\]
di mana $m$ adalah bilangan positif dan $C$ adalah kurva yang ditunjukkan
pada Gambar XXX. Jika kita membatasi pada setengah bidang atas sehingga
$y\geq0$, maka
\[
\left| e^{\imath mz} \right| = \left|e^{\imath mx}\right| \left| e^{-my} \right| = e^{-my} \leq 1
\]
sehingga
\[
\left|f(z)e^{\imath mz}\right|=\left|f(z)\right|\left|e^{\imath mz}\right|\leq\left|f(z)\right|
\]
Dengan menggunakan langkah yang sama seperti pada jenis integral sebelumnya,
diperoleh
\[
\int_{-\infty}^{\infty}f(x)e^{\imath mx}\ \mathrm{d}x=2\pi\imath\sum_{n=1}^{N}\left(b_{1}\right)_{n}
\]
di mana $b_{1}$ melibatkan residu dari $f(z)e^{\imath mz}$ pada
seluruh singularitas yang ada pada setengah bidang atas.

\rule[0.5ex]{1\columnwidth}{1pt}

Contoh

Hitung integral berikut di mana $C$ adalah lingkaran $|z|=2$.

\[
\oint_{C}\frac{\cos(z)}{z^{3}}\ \mathrm{d}z
\]

Ekspansi fungsi $\cos(z)$:
\[
\cos(z)=1-\frac{z^{2}}{2!}+\frac{z^{4}}{4!}-\cdots
\]
Integran menjadi:
\[
\frac{\cos(z)}{z^{3}}=\frac{1}{z^{3}}-\frac{1}{2z}+\frac{z}{4!}+\cdots
\]
Residu adalah koefisien dari suku $1/z$, sehingga
\[
b_{1}=-\frac{1}{2}
\]
Nilai dari integral adalah
\[
\oint_{C}\frac{\cos(z)}{z^{3}}\ \mathrm{d}z=2\pi\imath\left(-\frac{1}{2}\right)=-\pi\imath
\]

\rule[0.5ex]{1\columnwidth}{1pt}

Contoh

Hitung integral
\[
\oint_{C}\frac{\mathrm{d}z}{z^{2}+1}
\]

Faktorisasi dari integral
\[
\frac{1}{z^{2}+1}=\frac{1}{(z+\imath)(z-\imath)}
\]

Ada dua singularitas dalam kontur $C$. Residu pada tiap singularitas
adalah
\[
\left(b_{1}\right)_{z=\imath}=\left.(z-\imath)\frac{1}{(z+\imath)(z-\imath)}\right|_{z=\imath}=\frac{1}{2\imath}
\]

\[
\left(b_{1}\right)_{z=-\imath}=\left.(z+\imath)\frac{1}{(z+\imath)(z-\imath)}\right|_{z=-\imath}=-\frac{1}{2\imath}
\]
Nilai dari integral adalah
\[
\oint_{C}\frac{\mathrm{d}z}{z^{2}+1}=2\pi\imath\left(\frac{1}{2\imath}-\frac{1}{2\imath}\right)=0
\]

\rule[0.5ex]{1\columnwidth}{1pt}

Contoh

Hitung integral
\[
\oint_{C}\frac{z^{2}-2}{z(z-1)(z+4)}\ \mathrm{d}z
\]

Di dalam kontur yang ditinjau, ada dua singularitas, satu pada $z=0$
dan yang lain pada $z=1$. Residu pada tiap pole adalah
\[
\left(b_{1}\right)_{z=0}=\left.z\frac{z^{2}-2}{z(z-1)(z+4)}\right|_{z=0}=\frac{-2}{(-1)(4)}=\frac{1}{2}
\]

\[
\left(b_{1}\right)_{z=1}=\left.(z-1)\frac{z^{2}-2}{z(z-1)(z+4)}\right|_{z=1}=\frac{1-2}{(1)(1+4)}=-\frac{1}{5}
\]

Nilai integralnya adalah
\[
\oint_{C}\frac{z^{2}-2}{z(z-1)(z+4)}\ \mathrm{d}z=2\pi\imath\left(\frac{1}{2}-\frac{1}{5}\right)=\frac{3\pi\imath}{5}
\]

\rule[0.5ex]{1\columnwidth}{1pt}

Contoh

Hitung integral
\[
\oint_{C}\frac{z}{(z-1)^{3}(z+3)}\ \mathrm{d}z
\]

Untuk integran ini, terdapat sebuah pole dengan orde-3 di dalam kontur
$C$. Residu pada pole tersebut adalah
\begin{align*}
b_{1} & =\frac{1}{2!}\frac{\mathrm{d}^{2}}{\mathrm{d}z^{2}}\left[(z-1)^{3}\frac{z}{(z-1)^{3}(z+3)}\right]_{z=1}\\
 & =\frac{1}{2}\left(\frac{-3}{32}\right)=-\frac{3}{64}
\end{align*}
Sehingga nilai integralnya adalah
\[
\oint_{C}\frac{z}{(z-1)^{3}(z+3)}\ \mathrm{d}z=2\pi\imath\left(-\frac{3}{64}\right)\approx-0.2945\imath
\]

\rule[0.5ex]{1\columnwidth}{1pt}

Contoh

Hitung nilai integral:
\[
\int_{0}^{2\pi}\frac{\mathrm{d}\theta}{2+\cos(\theta)}
\]

Dengan menggunakan transformasi diperoleh
\[
\int_{0}^{2\pi}\frac{\mathrm{d}\theta}{2+\cos(\theta)}=\oint_{C}\frac{\mathrm{d}z/\imath z}{2+\dfrac{1}{2}\left(z+\dfrac{1}{z}\right)}=-2\imath\oint_{C}\frac{\mathrm{d}z}{z^{2}+4z+1}
\]
di mana $C$ adalah lingkaran satuan. Akar dari penyebut pada integran
adalah
\[
z=-2\pm\sqrt{3}
\]
atau $z_{1}=-2+\sqrt{3}\approx-0.2679$ dan $z_{2}=-2-\sqrt{3}\approx-3.732$.
Pole pertama berada pada lingkaran satuan sehingga kita harus menentukan
residu pada pole tersebut.
\[
\left(b_{1}\right)_{z_{1}}=\left.\frac{1}{z+2+\sqrt{3}}\right|_{z=z_{1}}\approx2.887
\]
sehingga
\[
\int_{0}^{2\pi}\frac{\mathrm{d}\theta}{2+\cos(\theta)}=-2\imath\left(2\pi b_{1}\right)\approx3.628
\]

\rule[0.5ex]{1\columnwidth}{1pt}

Contoh

Hitung nilai integral
\[
\int_{0}^{\infty}\frac{\mathrm{d}x}{1+x^{2}}
\]

Tinjau fungsi variabel kompleks $f(z)=1/(1+z^{2})$. Dua pole berada
pada titik di mana $1+z^{2}=0$, yaitu $z_{1}=\imath$ dan $z_{2}=-\imath$.
Pole pertama berada di setengah bidang atas. Residu pada pole tersebut
adalah
\[
\left(b_{1}\right)_{z_{1}}=\left.(z-\imath)\frac{1}{(z-\imath)(z+\imath)}\right|_{z=\imath}=\frac{1}{2\imath}
\]
Nilai dari integral
\[
\int_{-\infty}^{\infty}\frac{\mathrm{d}x}{1+x^{2}}=2\pi\imath\left(\frac{1}{2\imath}\right)=\pi
\]
Karena integran adalah fungsi genap, maka
\[
\int_{0}^{\infty}\frac{\mathrm{d}x}{1+x^{2}}=\frac{1}{2}\int_{-\infty}^{\infty}\frac{\mathrm{d}x}{1+x^{2}}=\frac{\pi}{2}
\]

\rule[0.5ex]{1\columnwidth}{1pt}

Contoh

Hitung nilai dari integral berikut
\[
\int_{-\infty}^{\infty}\frac{\cos(x)}{1+x^{2}}\ \mathrm{d}x
\]
\[
\int_{-\infty}^{\infty}\frac{\sin(x)}{1+x^{2}}\ \mathrm{d}x
\]

Dalam hal ini integral kompleks yang diperlukan adalah:
\[
I_{1}=\oint_{C}\frac{e^{\imath z}}{1+z^{2}}\ \mathrm{d}z
\]
Integran ini memiliki dua pole pada $z=\pm\imath$, yang salah satunya
berada di atas setengah bidang atas. Residu pada $z=\imath$ adalah
\[
\left(b_{1}\right)_{z=\imath}=\left.(z-\imath)\frac{e^{\imath z}}{1+z^{2}}\right|_{z=\imath}=\left.\frac{e^{\imath z}}{1+z}\right|_{z=\imath}=
\]
yang memberikan nilai integral
\[
\int_{-\infty}^{\infty}\frac{e^{\imath x}}{1+x^{2}}\ \mathrm{d}x=2\pi\imath\left(\frac{e^{-1}}{2\imath}\right)=\frac{\pi}{e}\approx1.1557
\]
Karena integral ini dapat dituliskan sebagai:
\[
\int_{-\infty}^{\infty}\frac{e^{\imath x}}{1+x^{2}}\ \mathrm{d}x=\int_{-\infty}^{\infty}\frac{\cos(x)}{1+x^{2}}\ \mathrm{d}x+\imath\int_{-\infty}^{\infty}\frac{\sin(x)}{1+x^{2}}\ \mathrm{d}x
\]
maka dengan mengevaluasi nilai real dan imajiner diperoleh bahwa:
\[
\int_{-\infty}^{\infty}\frac{\cos(x)}{1+x^{2}}\ \mathrm{d}x=\frac{\pi}{e}
\]
\[
\int_{-\infty}^{\infty}\frac{\sin(x)}{1+x^{2}}\ \mathrm{d}x=0
\]





\end{document}
