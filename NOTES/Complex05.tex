\input{PREAMBLE}

% -------------------------
\begin{document}

\title{%
{\small TF2201 Matematika Rekayasa II}\\
Variabel Kompleks: Ekspansi Deret
}
\author{Fadjar Fathurrahman\\
Teknik Fisika\\
Institut Teknologi Bandung}
\date{2023}
\maketitle


\section{Deret Taylor}

Suatu fungsi analitik dapat diekspansi ke dalam deret pangkat tak-hingga.
Jika $f(z)$ analitik pada titik $z=a$, maka $f(z)$ dapat dinyatakan dalam suatu
deret pangkat dari $(z-a)$.
Untuk menunjukkan hal ini, kita akan melakukan
ekspansi dari $(w-z)^{-1}$ sebagai berikut
\begin{align*}
\frac{1}{w-z} & =\frac{1}{(w-a)-(z-a)} = \frac{1}{w-a}\left[\frac{1}{1-\dfrac{z-a}{w-a}}\right]\\
 & =\frac{1}{w-a}\left[1 + \frac{z-a}{w-a} + \left(\frac{z-a}{w-a}\right)^{2}+\cdots+\left(\frac{z-a}{w-a}\right)^{n-1}+R_{n}(z,w)\right]
\end{align*}
di mana
\begin{equation*}
R_{n}(z,w)=\frac{1}{1-\dfrac{z-a}{w-a}}\left(\frac{z-a}{w-a}\right)^{n}
\end{equation*}
dan kita telah menggunakan identitas
\begin{equation}
\frac{1}{1-r}=1+r+r^{2}+\cdots+r^{n-1}+\frac{r^{n}}{1-r}
\end{equation}
dengan $r = \dfrac{z-a}{w-a}$.

Substitusi ekspansi ini pada Rumus Integral Cauchy
\begin{equation*}
f(z)=\frac{1}{2\pi\imath}\oint\frac{f(w)}{w-z}\ \mathrm{d}z
\end{equation*}
sehingga diperoleh
\begin{align*}
f(z) & = \frac{1}{2\pi\imath}\oint_{C}\frac{f(w)}{w-a}\left[1+\frac{z-a}{w-a}+\cdots+\left(\frac{z-a}{w-a}\right)^{n-1}+R_{n}(z,w)\right]\ \mathrm{d}w\\
 & =\frac{1}{2\pi\imath}\oint_{C}\frac{f(w)}{w-a}\ \mathrm{d}w+\frac{z-a}{2\pi i}\oint_{C}\frac{f(w)}{(w-a)^{2}}\ \mathrm{d}w+\cdots+\\
 & +\frac{(z-a)^{n-1}}{2\pi\imath}\oint_{C}\frac{f(w)}{(w-a)^{n}}\ \mathrm{d}w+\frac{(z-a)^{n}}{2\pi\imath}\oint_{C}\frac{f(w)}{(w-z)(w-a)^{n}}\ \mathrm{d}w
\end{align*}
%
di mana suku sisa dapat dituliskan menjadi:
%
\begin{equation*}
R_{n}(z,w)=\frac{w-a}{w-z}\left(\frac{z-a}{w-a}\right)^{n}
\end{equation*}
Dengan menggunakan Rumus Integral Cauchy untuk turunan fungsi, kita juga dapat menuliskan:
\begin{equation*}
f(z) = f(a) + \frac{1}{1!}f'(a)(z-a) + \cdots+\frac{1}{(n-1)!}f^{(n-1)}(a)(z-a)^{n-1} + 
\frac{(z-a)^{n}}{2\pi\imath} \oint_{C} \frac{f(w)}{(w-z)(w-a)^{n}}\ \mathrm{d}w
\end{equation*}

Integral pada persamaan ini adalah suku sisa (\emph{remainder term})
dan untuk nilai-nilai $z$, $w$, dan $a$, suku ini akan bernilai
nol ketika $n$ menuju tak hingga.

Misalkan $C$ adalah lingkaran yang berpusat pada $z=a$ dan $z$
adalah suatu titik di dalam lingkaran $C$. Jika $f(z)$ analitik
pada $C$ dan di dalam interiornya, yaitu $\left|z-a\right|=r$, dan
sehingga $r<R=\left|w-a\right|$. Misalkan $M=\max\left|f(z)\right|$
untuk $z$ pada $C$. Maka dengan teorema nilai maksimum
%
\begin{theorem}
Misalkan $\left| f(z) \right| \leq M$ di sepanjang kontur $C$ dan panjang
dari $C$ adalah $L$ maka:
\begin{equation*}
\left|\int_{C}f(z)\ \mathrm{d}z\right|\leq ML
\end{equation*}
\end{theorem}
%
dan $\left|w-z\right|\geq R-r$ dapat diperoleh
\begin{equation*}
\left|\frac{(z-a)^{n}}{2\pi\imath}\oint_{C}\frac{f(w)\ \mathrm{d}w}{(w-z)(w-a)^{n}}\right| \leq
\frac{r^{n}}{2\pi}\frac{M}{R-r}\frac{2\pi R}{R^{n}}=M\frac{R}{R-r}\left(\frac{r}{R}\right)^{n}
\end{equation*}
Karena $r<R$, $(r/R)^{n} \rightarrow 0$ untuk $n \rightarrow \infty$.

Dengan demikian kita telah membuktikan konvergensi dari Deret Taylor:
\begin{equation*}
f(z)=f(a)+f'(a)(z-a)+f''(a)\frac{z-a}{2!}+\cdots+f^{(n)}(a)\frac{z-a}{n!}+\cdots
\end{equation*}

Konvergensi ini berlaku pada lingkaran terbesar di sekitar $z=a$
yang di dalamnya $f(z)$ bersifat analitik. Jika $\left|z-a\right|=R$
berada di dalam lingkaran ini, maka $R$ adalah radius konvergensi
dan deret ini akan konvergen pada himpunan terbuka $\left|z-a\right|<R$.
Sebaliknya, deret ini akan divergen untuk $z$ yang memenuhi $\left|z-a\right|>R$.
Konvergensi pada lingkaran, $\left|z-a\right|=R$, lebih sulit untuk
dibuktikan, namun $f(z)$ akan memiliki titik singular di suatu titik
pada lingkaran $\left|z-a\right|=R$ oleh definisi dari $R$.

Untuk kasus daerah lingkaran di sekitar titik asal, $a=0$, diperoleh:
\[
f(z)=f(0)+f'(0)z+f''(z)\frac{z^{2}}{2!}+\cdots
\]
yang sering juga disebut sebagai Deret Maclaurin, terutama jika $z=x$
adalah bilangan real.

\begin{contoh}
Gunakan representasi Deret Taylor dari $f(z)$ untuk menemukan ekspansi
deret disekitar titik asal untuk (a) $f(z)=\sin(z)$, (b) $f(z)=e^{z}$,
(c) $f(z)=1/(1-z)^{m}$.
\end{contoh}

(a) Kita perlu mencari terlebih dahulu turunan dari fungsi pada $z=0$.

\begin{align*}
f'(0) & =\cos(0)=1\\
f''(0) & =-\sin(0)=0\\
f'''(0) & =-\cos(0)=-1
\end{align*}
sehingga diperoleh deret Taylor dari $f(z)=\sin(z)$ adalah:
\begin{align*}
\sin(z) & =\sin(0)+1\cdot(z-0)+0\cdot\frac{(z-0)^{2}}{2!}-1\cdot\frac{(z-0)^{3}}{3!}+\cdots\\
 & =z-\frac{z^{3}}{3!}+\frac{z^{5}}{5!}-\cdots
\end{align*}
Deret ini valid untuk semua $z$ karena tidak ada titik singular pada
bidang kompleks.

(b) Untuk $f(z)=e^{z}$, diperoleh:
\begin{align*}
f'(0) & =e^{0}=1\\
f''(0) & =e^{0}=1\\
f'''(0) & =e^{0}=1
\end{align*}
dan seterusnya.

Deret Taylor dari $f(z)=e^{z}$ adalah:
\begin{align*}
e^{z} & =e^{0}+1\cdot z+1\cdot\frac{z^{2}}{2!}+1\cdot\frac{z^{3}}{3!}+\cdots\\
 & =1+z+\frac{z^{2}}{2!}+\frac{z^{3}}{3!}+\cdots
\end{align*}
yang valid untuk seluruh $z$.

(c) Turunan yang diperlukan untuk mencari ekspansi Taylor dari $f(z)=\dfrac{1}{\left(1-z\right)^{m}}$
adalah:
\begin{align*}
f'(z) & =m(1-z)^{-m-1}\\
f''(z) & =m(m+1)(1-z)^{-m-2}\\
f'''(z) & =m(m+1)(m+2)(1-z)^{-m-3}
\end{align*}
dan seterusnya. Diperoleh deret Taylor sebagai berikut.
\begin{align*}
\frac{1}{(1-z)^{m}} & =\frac{1}{(1-0)^{m}}+m(1-0)^{-m-1}z+m(m+1)(1-0)^{-m-2}\frac{z^{2}}{2!}+\cdots\\
 & =1+mz+m(m+1)\frac{z^{2}}{2!}+m(m+1)(m+2)\frac{z^{3}}{3!}+\cdots
\end{align*}
Deret ini konvergen untuk $|z|<1$ dan tidak konvergen untuk $|z|\geq1$
karena terdapat titik singular pada $z=1$.

Untuk $m=1$ kita memperoleh:
\[
\frac{1}{1-z}=1+z+z^{2}+z^{3}+\cdots
\]

\begin{contoh}
Tentukan representasi deret Taylor dari $\ln(1+z)$ dengan menggunakan
\begin{equation*}
\frac{\mathrm{d}}{\mathrm{d}z}\ln(1+z)=\frac{1}{1+z}
\end{equation*}
\end{contoh}

Dengan menggunakan hasil dari soal sebelumnya, diperoleh
\[
\frac{1}{1+z}=\frac{1}{1-(-z)}=1-z+z^{2}-z^{3}+\cdots
\]
Dengan menggukan integrasi
\[
\int\mathrm{d}\left[\ln(1+z)\right]\ \mathrm{d}z=\int\frac{1}{1+z}\ \mathrm{d}z
\]
sehingga diperoleh
\[
\ln(1+z)=\int\frac{1}{1+z}\ \mathrm{d}z=z-\frac{z^{2}}{2}+\frac{z^{3}}{3}-\frac{z^{4}}{4}+\cdots+C
\]
Konstanta integrasi adalah $C=0$ karena ketika $z=0$ kita harus
memiliki $\ln(1)=0$. Sehingga ekspansi deret Taylor yang diperlukan
adalah:
\[
\ln(1+z)=z-\frac{z^{2}}{2}+\frac{z^{3}}{3}-\frac{z^{4}}{4}+\cdots
\]
Ekspansi ini valid untuk $|z|<1$ karena terdapat singularitas pada
$z=-1$.

\begin{contoh}
Tentukan deret Taylor dari
\[
f(z)=\frac{1}{z^{2}-3z+2}
\]
\end{contoh}

Dengan menggunakan pecahan parsial:
\[
\frac{1}{z^{2}-3z+2}=\frac{1}{(z-2)(z-1)}=\frac{1}{z-2}-\frac{1}{z-1}
\]

Representasi deret dari suku tersebut adalah
\[
\frac{1}{z-1}=-\frac{1}{1-z}=-(1+z+z^{2}+z^{3}+\cdots)
\]
dan
\begin{align*}
\frac{1}{z-2} & =-\frac{1}{2}\left(\frac{1}{1-z/2}\right)=-\frac{1}{2}\left[1+\frac{z}{2}+\left(\frac{z}{2}\right)^{2}+\left(\frac{z}{2}\right)^{3}+\cdots\right]\\
 & =-\frac{1}{2}\left[1+\frac{z}{2}+\frac{z^{2}}{4}+\frac{z^{3}}{8}+\cdots\right]
\end{align*}
sehingga diperoleh:
\[
\frac{1}{z^{2}-3z+2}=\frac{1}{2}+\frac{3}{4}z+\frac{7}{8}z^{2}+\frac{15}{16}z^{3}+\cdots
\]


\begin{contoh}
Tentukan ekspansi deret Taylor dari
\[
f(z)=\frac{1}{z^{2}-9}
\]
di sekitar titik $z=1$.
\end{contoh}

Dalam pecahan parsial:
\begin{align*}
\frac{1}{z^{2}-9} & =\frac{1}{(z-3)(z+3)}=\frac{1}{6}\left(\frac{1}{z-3}\right)-\frac{1}{6}\left(\frac{1}{z+3}\right)\\
 & =-\frac{1}{6}\left[\frac{1}{2-(z-1)}\right]-\frac{1}{6}\left[\frac{1}{4+(z-1)}\right]\\
 & =-\frac{1}{12}\left[\frac{1}{1-\dfrac{z-1}{2}}\right]-\frac{1}{24}\left[\frac{1}{1-\left(-\dfrac{z-1}{4}\right)}\right]
\end{align*}
yang dapat diekspansi menjadi deret Taylor
\begin{align*}
\frac{1}{z^{2}-9}= & -\frac{1}{12}\left[1+\frac{z-1}{2}+\left(\frac{z-1}{2}\right)^{2}+\left(\frac{z-1}{2}\right)^{3}+\cdots\right]\\
 & -\frac{1}{24}\left[1-\frac{z-1}{4}+\left(\frac{z-1}{4}\right)^{2}-\left(\frac{z-1}{4}\right)^{3}+\cdots\right]\\
= & -\frac{1}{8}-\frac{1}{32}(z-1)-\frac{3}{128}(z-1)^{2}-\frac{5}{512}(z-1)^{3}+\cdots
\end{align*}

Singularitas terdekat adalah titik $z=3$ sehingga radius konvergensi
adalah $2$, yaitu $|z-1|<2$.



\section{Deret Laurent}

Ada banyak aplikasi di mana kita ingin melakukan ekspansi suatu fungsi
$f(z)$ menjadi suatu deret di sekitar $z=a$, yang merupakan titik
singular. Tinjau daerah cincin (\textit{annulus}) pada Gambar \ref{fig:Potter_Ch10_Fig17}.
Fungsi $f(z)$
analitik pada daerah cincin, namun mungkin terdapat titik singular
di dalam lingkaran kecil atau di luar lingkaran besar. Kemungkinan
adanya titik singular di dalam lingkaran kecil menghalangi kita untuk
melakukan ekspansi deret Taylor karena fungsi $f(z)$ harus analitik
pada seluruh titik di dalam daerah interior. Kita dapat menggunakan
Rumus Integral Cauchy untuk membagi daerah seperti pada Gambar \ref{fig:Potter_Ch10_Fig17}.

\begin{figure}[h]
{\centering
\includegraphics[scale=1.0]{../images_priv/Potter_Ch10_Fig_17.png}
\par}
\caption{Daerah cincin atau annulus yang di dalamnya terdapat titik singular.}
\label{fig:Potter_Ch10_Fig17}
\end{figure}

Dengan menggunakan Rumus Integral Cauchy diperoleh
\begin{align*}
f(z) & =\frac{1}{2\pi\imath}\oint_{C'}\frac{f(w)}{w-z}\ \mathrm{d}w\\
 & =\frac{1}{2\pi\imath}\oint_{C_{2}}\frac{f(w)}{w-z}\ \mathrm{d}w-\frac{1}{2\pi\imath}\oint_{C_{1}}\frac{f(w)}{w-z}\ \mathrm{d}w
\end{align*}
di mana kedua kontur $C_{1}$dan $C_{2}$ sama-sama dilalui berlawanan
arah jarum jam. Tanda negatif diperoleh dari arah integrasi yang berlawanan
pada $C_{1}$. Sekaran kita akan mengekspansi $(w-z)^{-1}$ pada integran
dalam suatu bentuk yang akan terdiri dari pangkat positif dari $(z-a)$
pada $C_{2}$ dan pangkat negatif untuk $C_{1}$. Jika tidak ada titik
singular pada $C_{1}$, maka koefisien dari pangkat negatif akan bernilai
nol dan Deret Taylor akan dihasilkan.
\begin{equation}
f(z) = \frac{1}{2\pi\imath} \oint_{C_{2}}\frac{f(w)}{w-a}\left[\frac{1}{1-\dfrac{z-a}{w-a}}\right]\ \mathrm{d}w +
\frac{1}{2\pi\imath} \oint_{C_{1}} \frac{f(w)}{z-a} \left[\frac{1}{1-\dfrac{w-a}{z-a}}\right]\ \mathrm{d}w
\label{eq:Potter_Eq_10_9_2}
\end{equation}
Dengan menggunakan argumen yang sama untuk membuktikan konvergensi
dari Deret Taylor, dapat ditunjukkan bahwa Persamaan \ref{eq:Potter_Eq_10_9_2} memberikan
\begin{align*}
f(z) & =a_{0}+a_{1}(z-a)+a_{2}(z-a)^{2}+\cdots+\\
 & +b_{1}(z-a)^{-1}+b_{2}(z-a)^{-2}+\cdots
\end{align*}
di mana
\begin{align*}
a_{n} & =\frac{1}{2\pi\imath}\oint_{C_{2}}\frac{f(w)}{(w-a)^{n+1}}\ \mathrm{d}w\\
b_{n} & =\frac{1}{2\pi\imath}\oint_{C_{1}}f(w)(w-a)^{n-1}\ \mathrm{d}w
\end{align*}
Deret yang dihasilkan ini dikenal dengan nama Deret Laurent.

Ekspresi integral untuk koefisien $a_{n}$ mirip dengan rumus untuk
turunan dari $f(z)$, namun ini hanya superfisial, karena $f(z)$
mungkin tidak terdefinisi pada $z=a$ dan $f(z)$ mungkin tidak analitik
pada titik tersebut. Perhatikan bahwa, jika $f(z)$ analitik pada
lingkaran $C_{1}$, integran untuk koefisien $b_{n}$ akan analitik
pada bidang kompleks, menyebabkan seluruh koefisien $b_{n}$ menjadi
nol, berdasarkan Teorema Integral Cauchy. Dalam hal ini, deret Laurent
tereduksi menjadi deret Taylor.

Ekspresi integral untuk koefisien deret Laurent biasanya tidak digunakan
secara langsung untuk menemukan koefisien. Karena ekspansi deret ini
unik, teknik elementer biasanya digunakan untuk menemukan ekspansi
deret Laurent. Daerah konvergensi dapat ditemukan, dalam banyak kasus,
dengan cara memanipulasi $f(z)$ ke dalam bentuk $1/(1-w)$ sehingga
$|w|<1$ memberikan daerah konvergensi.

Karena $|w| < 1$, kita memperoleh deret geometrik:
\begin{equation*}
\frac{1}{1-w} = 1 + w + w^{2} + w^{3} + \cdots
\end{equation*}
yang merupakan ekspansi deret Laurent di sekitar $w=0$.

\begin{contoh}
Tentukan ekspansi deret Laurent dari
\[
f(z)=\frac{1}{z^{2}-3z+2}
\]
yang valid pada daerah-daerah berikut.

{\centering
\includegraphics[scale=1.0]{../images_priv/Potter_Ch10_Fig_Ex_9_1.png}
\par}

\end{contoh}

\textbf{Daerah (a)} Pada daerah ini kita melakukan ekspansi di sekitar titik pusat.
\begin{align*}
\frac{1}{z^{2}-3z+2} & =\frac{1}{(z-2)(z-1)}=\frac{1}{z-2}-\frac{1}{z-1}\\
 & =-\frac{1}{2}\left(\frac{1}{1-z/2}\right)-\frac{1}{z}\left(\frac{1}{1-1/z}\right)
\end{align*}
Pecahan pertama memiliki singularitas pada $z/2=1$ atau $z=2$, dan
dapat diekspansi menjadi deret Taylor yang konvergen jika $|z|<2$.

Pecahan kedua memiliki singularitas pada $1/z=1$ dan dapat diekspansi
menjadi deret Laurent yang konvergen jika $|1/z|<1$ atau $|z|>1$.

Dua pecahan tersebut dinyatakan dalam deret-deret berikut.
\begin{align*}
-\frac{1}{2}\left(\frac{1}{1-z/2}\right) & =-\frac{1}{2}\left[1+\frac{z}{2}+\left(\frac{z}{2}\right)^{2}+\left(\frac{z}{2}\right)^{3}+\cdots\right]\\
 & =-\frac{1}{2}-\frac{z}{4}-\frac{z^{2}}{8}-\frac{z^{3}}{16}-\cdots
\end{align*}
yang valid untuk $|z|<2$

\begin{align*}
-\frac{1}{z}\left(\frac{1}{1-1/z}\right) & =-\frac{1}{z}\left[1+\frac{1}{z}+\left(\frac{1}{z}\right)^{2}+\left(\frac{1}{z}\right)^{3}+\cdots\right]\\
 & =-\frac{1}{z}-\frac{1}{z^{2}}-\frac{1}{z^{3}}-\frac{1}{z^{4}}-\cdots
\end{align*}
yang valid untuk $|z|>1$.

Dengan menjumlahkan kedua deret tersebut diperoleh:
\[
\frac{1}{z^{2}-3z+2}=\cdots-\frac{1}{z^{3}}-\frac{1}{z^{2}}-\frac{1}{z}-\frac{1}{2}-\frac{z}{4}-\frac{z^{2}}{8}-\frac{z^{3}}{16}-\cdots
\]
yang valid untuk daerah $1<|z|<2$.

\textbf{Daerah (b)} Pada daerah di luar lingkaran $|z|=2$, untuk $1/(z-1)$ digunakan
ekspansi yang sama seperti sebelumnya
\[
\frac{1}{z-1}=\frac{1}{z}\left(\frac{1}{1-1/z}\right)=\frac{1}{z}+\frac{1}{z^{2}}+\frac{1}{z^{3}}+\cdots
\]
yang valid untuk $|1/z|<1$ atau $|z|>1$.

Untuk $1/(z-2)$ digunakan:
\begin{align*}
\frac{1}{z-2} & =\frac{1}{z}\left(\frac{1}{1-2/z}\right)=\frac{1}{z}\left[1+\frac{2}{z}+\left(\frac{2}{z}\right)^{2}+\left(\frac{2}{z}\right)^{3}+\cdots\right]\\
 & =\frac{1}{z}+\frac{2}{z^{2}}+\frac{4}{z^{3}}+\frac{8}{z^{4}}+\cdots
\end{align*}
yang valid untuk $|2/z|<1$ atau $|z|>2$.

Dengan menjumlahkan keduanya diperoleh
\[
\frac{1}{z^{2}-3z+2}=\frac{1}{z^{2}}+\frac{3}{z^{3}}+\frac{7}{z^{4}}+\frac{15}{z^{5}}+\cdots
\]
yang valid untuk $|z|>2$.

\textbf{Daerah (c)} Untuk mendapatkan ekspansi deret pada daerah lingkaran $0<|z-1|<1$,
dilakukan ekspansi di sekitar titik $z=1$ dan diperoleh
\begin{align*}
\frac{1}{z^{2}-3z+2} & =\frac{1}{z-1}\left(-\frac{1}{2-z}\right)=\frac{1}{z-1}\left(\frac{-1}{1-(z-1)}\right)\\
 & =\frac{-1}{z-1}\left[1+(z-1)+(z-1)^{2}+(z-1)^{3}+\cdots\right]\\
 & =-\frac{1}{z-1}-1-(z-1)-(z-1)^{2}-\cdots
\end{align*}
yang valid untuk $0<|z-1|<1$.





\end{document}
