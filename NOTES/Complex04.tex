\documentclass[a4paper,10pt,bahasa]{extarticle} % screen setting
\usepackage[a4paper]{geometry}

%\documentclass[b5paper,11pt,bahasa]{article} % screen setting
%\usepackage[b5paper]{geometry}

%\geometry{verbose,tmargin=1.5cm,bmargin=1.5cm,lmargin=1.5cm,rmargin=1.5cm}

\geometry{verbose,tmargin=2.0cm,bmargin=2.0cm,lmargin=2.0cm,rmargin=2.0cm}

\setlength{\parskip}{\smallskipamount}
\setlength{\parindent}{0pt}

%\usepackage{cmbright}
%\renewcommand{\familydefault}{\sfdefault}

\usepackage{amsmath}
\usepackage{amssymb}

%\usepackage[libertine]{newtxmath}
%\usepackage[no-math]{fontspec}
%\setmainfont{Linux Libertine O}

\usepackage{fontspec}
\usepackage{lmodern}
\setmonofont{JuliaMono-Regular}


\usepackage{hyperref}
\usepackage{url}
\usepackage{xcolor}
\usepackage{enumitem}
\usepackage{mhchem}
\usepackage{graphicx}
\usepackage{float}

\usepackage{minted}

\newminted{julia}{breaklines,fontsize=\footnotesize,linenos}
\newminted{python}{breaklines,fontsize=\footnotesize,linenos}

\newminted{pycon}{breaklines,fontsize=\footnotesize}

\newminted{bash}{breaklines,fontsize=\footnotesize}
\newminted{text}{breaklines,fontsize=\footnotesize}

\newcommand{\txtinline}[1]{\mintinline[breaklines,fontsize=\footnotesize]{text}{#1}}
\newcommand{\jlinline}[1]{\mintinline[breaklines,fontsize=\footnotesize]{julia}{#1}}
\newcommand{\pyinline}[1]{\mintinline[breaklines,fontsize=\footnotesize]{python}{#1}}

\newmintedfile[juliafile]{julia}{breaklines,fontsize=\footnotesize}
\newmintedfile[pythonfile]{python}{breaklines,fontsize=\footnotesize}
\newmintedfile[fortranfile]{fortran}{breaklines,fontsize=\footnotesize}

\usepackage{mdframed}
\usepackage{setspace}
\onehalfspacing

\usepackage{babel}
\usepackage{appendix}

\newcommand{\highlighteq}[1]{\colorbox{blue!25}{$\displaystyle#1$}}
\newcommand{\highlight}[1]{\colorbox{red!25}{#1}}

\newtheorem{theorem}{Teorema}
\definecolor{thmbgcol}{rgb}{0.9,0.9,0.9}
\BeforeBeginEnvironment{theorem}{
    \begin{mdframed}[backgroundcolor=thmbgcol,%
        topline=false,bottomline=false,%
        leftline=false,rightline=false]
}
\AfterEndEnvironment{theorem}{\end{mdframed}}



\newcounter{soal}%[section]
\newenvironment{soal}[1][]{\refstepcounter{soal}\par\medskip
   \noindent \textbf{Soal~\thesoal. #1} \sffamily}{\medskip}


\definecolor{mintedbg}{rgb}{0.9,0.9,0.9}
\BeforeBeginEnvironment{minted}{
    \begin{mdframed}[backgroundcolor=mintedbg,%
        topline=false,bottomline=false,%
        leftline=false,rightline=false]
}
\AfterEndEnvironment{minted}{\end{mdframed}}


\BeforeBeginEnvironment{soal}{
    \begin{mdframed}[%
        topline=true,bottomline=false,%
        leftline=true,rightline=false]
}
\AfterEndEnvironment{soal}{\end{mdframed}}



\newcounter{contoh}%[section]
\newenvironment{contoh}[1][]{\refstepcounter{soal}\par\medskip
   \noindent \textbf{Contoh~\thesoal. #1} \rmfamily}{\medskip}

\BeforeBeginEnvironment{contoh}{
    \begin{mdframed}[%
        topline=true,bottomline=false,%
        leftline=true,rightline=false]
}
\AfterEndEnvironment{contoh}{\end{mdframed}}


\renewcommand{\imath}{\mathrm{i}}

% -------------------------
\begin{document}

\title{%
{\small TF2201 Matematika Rekayasa II}\\
Variabel Kompleks: Integrasi Fungsi Kompleks
}
\author{Fadjar Fathurrahman\\
Teknik Fisika\\
Institut Teknologi Bandung}
\date{2023}
\maketitle


\section{Busur (\textit{arc}) dan kontur (\textit{contour})}

Suatu busur halus (smooth arc) adalah himpunan titik-titik $(x,y)$
yang memenuhi:
\begin{equation*}
x=\phi(t),\ \ \ \ y=\psi(t),\ \ \ \ a\leq t\leq b
\end{equation*}
di mana $\phi'(t)$ dan $\psi'(t)$ bersifat kontinu pada $[a,b]$
dan tidak bernilai nol pada $t$ yang sama.

Misalnya lingkaran $x^{2} + y^{2} = 1$ dapat direpresentasikan secara
parametrik dengan
\begin{equation*}
x=\cos(t),\ \ \ \ y=\sin(t),\ \ \ \ 0 \leq t \leq 2\pi
\end{equation*}

Contoh lain, representasi parametrik
\begin{equation*}
x=t,\ \ \ \ y=t^{2},\ \ \ \ -\infty<t<\infty    
\end{equation*}
mendefinisikan parabola $y=x^{2}$.

Perhatikan bahwa suatu representasi parametrik memberikan suatu urutan
untuk titik-titik pada busur.

Suatu busur halus memiliki panjang yang diberikan oleh:
\begin{equation*}
L = \int_{a}^{b} \sqrt{\left[\phi'(t)\right]^{2}+\left[\psi'(t)\right]^{2}}\ \mathrm{d}t    
\end{equation*}

Sebuah kontur adalah sekumpulan busur halus yang saling terhubung
secara kontinu.

Suatu kontur tertutup sederhana atau kurva Jordan, adalah suatu kontur
yang tidak berpotongan dengan dirinya sendiri kecuali pada $\phi(a)=\phi(b)$
dan $\psi(a)=\psi(b)$. Suatu kontur tertutup sederhana membagi sebuah
bidang menjadi dua bagian, yaitu bagian "dalam" dan bagian "luar",
dan dikatakan dilalui secara positif jika bagian dalam berada pada
sebelah kiri. Contoh kurva dapat dilihat pada Gambar
\ref{fig:Potter_Fig_10_05}.

\begin{figure}[h]
{\centering
\includegraphics[scale=1.0]{../images_priv/Potter_Ch10_Fig_05.png}
\par}
\caption{Beberapa contoh kurva}\label{fig:Potter_Fig_10_05}
\end{figure}

Lingkaran pada bidang kompleks memiliki representasi yang sederhana,
menggunakan bentuk polar dan eksponensial dari $z$. Suatu lingkaran
$|z|=a$ diberikan secara parametrik dengan persamaan:
\begin{equation*}
z=a\cos(\theta)+\imath a\sin(\theta),\ \ \ \ 0\leq\theta\leq2\pi
\end{equation*}
atau:
\begin{equation*}
z=ae^{\imath\theta},\ \ \ \ 0\leq\theta\leq2\pi
\end{equation*}

\begin{figure}[h]
{\centering
\includegraphics[scale=1.0]{../images_priv/Potter_Ch10_Fig_06.png}
\par}
\caption{Kontur lingkaran}\label{fig:Potter_Fig_10_06}
\end{figure}

Suatu lingkaran dengan jari-jari $a$ berpusat pada $z_{0}$,
seperti ditunjukkan pada Gambar \ref{fig:Potter_Fig_10_06},
dapat dinyatakan dengan persamaan:
\begin{equation*}
z-z_{0}  = ae^{\imath\theta} = a\cos(\theta) + \imath a\sin(\theta)
\end{equation*}
dengan $\theta$ diukur dari suatu sumbu yang melewati titik $z=z_{0}$
dan sejajar dengan sumbu $x$.

\section{Integral garis}

Misalkan $z_{0}$ dan $z_{1}$ adalah dua titik pada bidang kompleks
dan $C$ adalah suatu kontur yang menghubungkan antara keduanya, misalkan
$C$ didefinisikan secara parametrik dengan
\begin{equation*}
x=\phi(t)\ \ \ \ y=\psi(t)
\end{equation*}
sehingga
\begin{equation*}
z_{0}=\phi(a)+\imath\psi(a),\ \ \ \ z_{1}=\phi(b)+\imath\psi(b)
\end{equation*}

Integral garis
\begin{equation*}
\int_{C}f(z)\ \mathrm{d}z=\int_{z_{0}}^{z_{1}}f(z)\ \mathrm{d}z
\end{equation*}
didefinisikan sebagai integral real:
\begin{align}
\int_{C}f(z)\ \mathrm{d}z & = \int_{C}(u + \imath v)(\mathrm{d}x+\imath\ \mathrm{d}y) \nonumber \\
 & =\int_{C}(u\ \mathrm{d}x - v \ \mathrm{d}y) +
 \imath\int_{C}(v\ \mathrm{d}x + u\ \mathrm{d}y) \label{eq:integ_line}
\end{align}
di mana kita telah menuliskan $f(z) = u + \imath v$.

Dari definisi ini dapat diperoleh beberapa hubungan sebagai berikut.
Pertama:
\begin{equation}
\int_{z_{0}}^{z_{1}}f(z)\ \mathrm{d}z=-\int_{z_{1}}^{z_{0}}f(z)\ \mathrm{d}z
\end{equation}
di mana jalur yang ditempuh pada integral pada RHS sama dengan LHS,
namun dengan arah yang berbeda.
Kedua:
\begin{equation}
\int_{z_{0}}^{z_{1}}kf(z)\ \mathrm{d}z = k \int_{z_{0}}^{z_{1}}f(z)\ \mathrm{d}z
\end{equation}

Jika kontur $C$ terdiri dari kumpulan kontur-kontur $C_{1},C_{2},\ldots,C_{k}$
yang terhubung secara kontinu maka:
\begin{equation*}
\int_{z_{0}}^{z_{1}}f(z)\ \mathrm{d}z =
\int_{C_{1}}f(z)\ \mathrm{d}z + \int_{C_{2}}f(z)\ \mathrm{d}z + \cdots +
\int_{C_{k}}f(z)\ \mathrm{d}z
\end{equation*}

\begin{theorem}
Misalkan $\left| f(z) \right| \leq M$ di sepanjang kontur $C$ dan panjang
dari $C$ adalah $L$ maka:
\begin{equation*}
\left|\int_{C}f(z)\ \mathrm{d}z\right|\leq ML
\end{equation*}
\end{theorem}



\begin{contoh}
Tentukan nilai dari:
\begin{equation*}
\int_{0}^{1 + \imath} z^{2}\ \mathrm{d}z    
\end{equation*}
di sepanjang kontur berikut. (a) garis lurus dari $0$ sampai $1+\imath$,
(b) garis poligon dari $0$ ke $1$ dan dari $1$ ke $1+\imath$.
\end{contoh}

Kontur diberikan pada gambar berikut.

{\centering
\includegraphics[scale=1.0]{../images_priv/Potter_Ch10_Fig_Ex_5_1.png}
\par}

Pada sembarang kontur dapat dituliskan:
\begin{align*}
\int_{0}^{1+\imath}z^{2}\ \mathrm{d}z & = \int_{0}^{1 + \imath}
\left[ \left( x^{2}-y^{2} \right) + \imath 2xy \right]
\left( \mathrm{d}x +\imath\ \mathrm{d}y \right) \\
 & =\int_{0}^{1 + \imath} \left[ \left( x^{2}-y^{2} \right)\ \mathrm{d}x -
 2xy\ \mathrm{d}y \right] + \imath \int_{0}^{1 + \imath}
 \left[ 2xy\ \mathrm{d}x + \left(x^{2}-y^{2} \right)\ \mathrm{d}y\right]
\end{align*}

(a) Kontur $C_{1}$: garis lurus dari $0$ ke $1+\imath$ dengan bentuk
parametrik:
\[
x=t,\ \ \ \ y=t,\ \ \ \ 0\leq t\leq1
\]
dengan $\mathrm{d}t=\mathrm{d}x=\mathrm{d}y$. Dengan substitusi parametrik
tersebut diperoleh:
\begin{align*}
\int_{0}^{1+\imath}z^{2}\ \mathrm{d}z & =\int_{0}^{1}\left[\left(t^{2}-t^{2}\right)\ \mathrm{d}t-2(t)(t)\ \mathrm{d}t\right]+\imath\int_{0}^{1}\left[2(t)(t)\ \mathrm{d}t+\left(t^{2}-t^{2}\right)\ \mathrm{d}t\right]\\
 & =\int_{0}^{1}\left[-2t^{2}\ \mathrm{d}t\right]+\imath\int_{0}^{1}\left[2t^{2}\ \mathrm{d}t\right]\\
 & =\left[-\frac{2}{3}t^{3}\right]_{0}^{1} + \imath\left[\frac{2}{3}t^{3}\right]_{0}^{1}\\
 & =\left(-\frac{2}{3}-0\right) + \imath\left(\frac{2}{3}\right)\\
 & =-\frac{2}{3}+\frac{2}{3}\imath
\end{align*}

(b) Kontur $C_{2}$ dengan $z=x$ dan kontur $C_{3}$ dengan $z=1+\imath y$:
\begin{equation*}
\int_{0}^{1+\imath}z^{2}\ \mathrm{d}z = \int_{0}^{1}x^{2}\ \mathrm{d}x +
\int_{0}^{1}\left(1 + \imath y\right)^{2} \imath\ \mathrm{d}y
\end{equation*}

....


\begin{contoh}
Tentukan nilai dari:
\begin{equation*}
\oint_{C} \frac{1}{z} \, \mathrm{d}z    
\end{equation*}
di mana $C$ adalah lingkaran dengan jari-jari satu yang berpusat
pada titik (0,0) (lingkaran ini sering juga disebut sebagai lingkaran satuan).
\end{contoh}

Representasi paling sederhana dari lingkaran ini adalah bentuk
eksponensial $z = e^{\imath \theta}$, sehingga
$\mathrm{d}z = \imath e^{\imath \theta}\, \mathrm{d}\theta$.
Kita dapat memperoleh
\begin{equation*}
\oint \frac{1}{z} \, \mathrm{d}z = \int_{0}^{2\pi}
\frac{\imath e^{\imath \theta}}{e^{\imath \theta}} \, \mathrm{d}\theta =
\imath \int_{0}^{2\pi} \mathrm{d}\theta = 2\pi \imath 
\end{equation*}

\begin{soal}
Ulangi contoh sebelumnya, namun dengan menggunakan lingkaran dengan jari-jari sembarang,
misalkan $r$, dengan $r$ adalah bilangan real positif.
\end{soal}


\begin{contoh}
Hitung
\begin{equation*}
\oint_{C} \frac{1}{z^n} \, \mathrm{d}z    
\end{equation*}
di mana $C$ adalah lingkaran satuan dan $n$ merupakan bilangan bulat positif
lebih besar dari 1.
\end{contoh}

Menggunakan $z = e^{\imath \theta}$ dan
$\mathrm{d}z = \imath e^{\imath \theta}\, \mathrm{d}\theta$, diperoleh:

\begin{align*}
\oint \frac{1}{z^n} \, \mathrm{d}z & = \int_{0}^{2\pi} \frac{\imath e^{\imath \theta}}{e^{n \imath \theta}} \\
& = \imath \int_{0}^{2\pi} e^{\imath \theta (1 - n)} \, \mathrm{d} \theta \\
& = \cdots
\end{align*}


\section{Teorema Green}

\begin{theorem}
Misalkan $C$ adalah kontur tertutup sederhana yang dilalui pada arah
positif dan membatasi daerah $R$. Misalkan juga $u$ dan $v$ adalah
kontinu dengan turunan parsial pertama yang juga kontinu pada $R$.
Maka
\begin{equation*}
\oint_{C}\left(u\ \mathrm{d}x-v\ \mathrm{d}y\right) =
-\iint_{R}\left(\frac{\partial v}{\partial x} +
\frac{\partial u}{\partial y}\right)\ \mathrm{d}x\mathrm{\ d}y
\end{equation*}
\end{theorem}


Tinjau kurva $C$ yang meliputi daerah $R$ pada
Gambar \ref{fig:Potter_Ch10_Fig_10}.
\begin{figure}[h]
{\centering
\includegraphics[scale=1.0]{../images_priv/Potter_Ch10_Fig_10.png}
\par}\label{fig:Potter_Ch10_Fig_10}
\end{figure}

\begin{align*}
\iint\frac{\partial v}{\partial x}\ \mathrm{d}x\mathrm{\ d}y & =\int_{h_{1}}^{h_{2}}\int_{x_{1}(y)}^{x_{2}(y)}\frac{\partial v}{\partial x}\ \ dx\mathrm{\ d}y\\
 & =\int_{h_{1}}^{h_{2}}\left[v(x_{2},y)-v(x_{1},y)\right]\ \mathrm{d}y\\
 & =\int_{h_{1}}^{h_{2}}v(x_{2},y)\ \mathrm{d}y+\int_{h_{2}}^{h_{1}}v(x_{1},y)\ \mathrm{d}y
\end{align*}

Integral pertama pada RHS adalah integral garis dari $v(x,y)$ pada
jalur ABC dari A ke C dan integral kedua adalah integral garis dari
$v(x,y)$ pada jalur ADC dari C ke A. Perhatikan bahwa daerah $R$
berada pada bagian kiri dari kurva. Kemudian
\begin{equation*}
\iint_{R}\frac{\partial v}{\partial x}\ \mathrm{d}x\ \mathrm{d}y=\oint_{C}v(x,y)\ \mathrm{d}y
\end{equation*}

Dengan cara yang sama dapat ditunjukkan bahwa:
\begin{equation*}
\iint_{R}\frac{\partial u}{\partial y}\ \mathrm{d}x\ \mathrm{d}y=-\oint_{C}u(x,y)\ \mathrm{d}y    
\end{equation*}
dan Teorema Green terbukti.




Tinjau integral
\begin{equation*}
\oint_{C}f(z)\ \mathrm{d}z
\end{equation*}
di mana $f(z)$ adalah fungsi analitik di dalam suatu daerah terhubung
sederhana $R$ yang dilingkupi oleh suatu kontur tertutup sederhana
$C$. Dapat dituliskan:
\begin{equation*}
\oint_{C}f(z)\ \mathrm{d}z =
\oint_{C} \left( u\ \mathrm{d}x-v\ \mathrm{d}y \right) +
\imath \oint_{C} \left( v\ \mathrm{d}x + u\ \mathrm{d}y \right)
\end{equation*}
Dengan menggunakan teorema Green:
\begin{equation*}
\oint_{C}f(z)\ \mathrm{d}z = -\iint_{R} \left(\frac{\partial v}{\partial x} +
\frac{\partial u}{\partial y}\right)\ \mathrm{d}x\ \mathrm{d}y -
\imath\iint_{R}\left(-\frac{\partial u}{\partial x} +
\frac{\partial v}{\partial y}\right)\ \mathrm{d}x\ \mathrm{d}y
\end{equation*}
Dengan menggunakan Persamaan Cauchy-Riemann, kita mendapatkan Teorema
Integral Cauchy:
\begin{equation*}
\oint_{C} f(z)\ \mathrm{d}z = 0
\end{equation*}

Dalam bentuk teorema

Misalkan $C$ adalah kontur tertutup sederhana yang melingkupi daerah
$R$ di mana $f(z)$ merupakan fungsi analitik. Maka:
\begin{equation*}
\oint_{C}f(z)\ \mathrm{d}z = 0
\end{equation*}

Suatu integral tak-tentu:
\begin{equation*}
F(z)=\int_{z_{0}}^{z}f(w)\ \mathrm{d}w
\end{equation*}
mendefinisikan $F$ sebagai fungsi dari $z$ sepanjang kontur yang
menghubungkan antara $z_{0}$ dan $z$ berada seluruhnya pada suatu
daerah $D$ yang terhubung sederhana dan di dalamnya $f(z)$ bersifat
analitik.

Selain itu:
\begin{equation*}
F'(z)=f(z)\ \ \ \ \text{pada}\ D
\end{equation*}

\rule[0.5ex]{1\columnwidth}{1pt}

Contoh

Evaluasi integral
\begin{equation*}
\int_{0}^{1 + \imath}z^{2}\ \mathrm{d}z
\end{equation*}

Misalkan:
\begin{equation*}
F(z)=\frac{z^{3}}{3}
\end{equation*}
Karena $F'(z) = z^{2}$, maka:
\begin{align*}
\int_{0}^{1+\imath}z^{2}\ \mathrm{d}z & =F(1+\imath)-F(0)\\
 & =\frac{(1+\imath)^{3}}{3}-0\\
 & =-\frac{2}{3}+\frac{2}{3}\imath
\end{align*}

\rule[0.5ex]{1\columnwidth}{1pt}

Kontur ekuivalen

Teorema integral Cauchy memungkinkan kita untuk mengganti integral
pada suatu kontur tertutup sederhana dengan integral lain pada suatu
kontur lain yang lebih sederhana, biasanya suatu lingkaran.

Tinjau integral $\int_{C_{1}}f(z)\ \mathrm{d}z$, di mana kontur $C_{1}$
ditunjukkan pada Gambar \ref{fig:Potter_Fig_10_13}.

\begin{figure}[h]
{\centering
\includegraphics[scale=1.0]{../images_priv/Potter_Ch10_Fig_13.png}
\par}
\caption{Kontur}\label{fig:Potter_Fig_10_13}
\end{figure}

Kita dapat mengatakan bahwa suatu kontur lain $C_{2}$ adalah kontur
ekuivalen dari $C_{1}$ jika:
\begin{equation*}
\oint_{C_{1}}f(z)\ \mathrm{d}z=\oint_{C_{2}}f(z)\ \mathrm{d}z
\end{equation*}
Pertanyaan selanjutnya adalah pada kondisi apa kontur $C_{1}$ dan
$C_{2}$ merupakan kontur ekuivalen.

Misalkan $f(z)$ adalah analitik pada daerah yang diapit oleh kontur
$C_{1}$ dan $C_{2}$ dan pada kontur tersebut seperti pada Gambar
\ref{fig:Potter_Fig_10_13}.

Misalkan segmen garis $C_{3}$ menghubungkan antara $C_{1}$ dan $C_{2}$
dan misalkan kontur $C$ dibentuk dari $C_{1}$ (berlawanan arah jarum
jam), $C_{3}$ ke $C_{2}$, $C_{2}$ (searah jarum jam), dan $C_{3}$
dari $C_{2}$ ke $C_{1}$. Berdasarkan teorema integral Cauchy:
\begin{equation*}
\oint_{C}f(z)\ \mathrm{d}z=0
\end{equation*}

Akan tetapi, dari konstruksi kontur $C$:
\begin{equation*}
\oint_{C}f(z)\ \mathrm{d}z = \oint_{C_{1}}f(z)\ \mathrm{d}z +
\oint_{C_{3}}f(z)\ \mathrm{d}z - \oint_{C_{2}}f(z)\ \mathrm{d}z - 
\oint_{C_{3}}f(z)\ \mathrm{d}z
\end{equation*}
Dari konstruksi di atas, kita melihat bahwa $C_{2}$ ekuivalen dengan
$C_{1}$ karena dua integral pada $C_{3}$ saling menghilangkan.







\end{document}
