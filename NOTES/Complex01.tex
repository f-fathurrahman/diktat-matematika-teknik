\input{PREAMBLE}

% -------------------------
\begin{document}

\title{%
{\small TF2201 Matematika Rekayasa II}\\
Variabel Kompleks: Representasi dan Aritmatika Dasar
}
\author{Fadjar Fathurrahman\\
Teknik Fisika\\
Institut Teknologi Bandung}
\date{2023}
\maketitle

%\tableofcontents

\section{Representasi bilangan kompleks}

Suatu variabel atau bilangan kompleks $z$ dapat ditulis dalam bentuk
\begin{equation}
z = x + \mathrm{i}y
\label{eq:complex-cart}
\end{equation}
dengan $x$ dan $y$ adalah bilangan real dan
\begin{equation}
\mathrm{i} = \sqrt{-1}
\end{equation}
Bilangan kompleks $z$ dapat diinterpretasikan secara geometrik geometrik
sebagai titik $(x,y)$ pada bidang Cartesian $xy$.
Sumbu-$x$ disebut juga dengan sumbu real dan sumbu-$y$
disebut sebagai sumbu imajiner.
Bentuk \ref{eq:complex-cart} dikenal juga sebagai bentuk Cartesian
dari bilangan kompleks.

Bagian real dari bilangan kompleks $z$ dapat ditulis dengan notasi
\begin{equation*}
\mathrm{Re}\ z=x
\end{equation*}
dan bagian imajiner dengan
\begin{equation*}
\mathrm{Im}\ z=y
\end{equation*}

Setiap bilagan kompleks memiliki konjugat yang dituliskan sebagain:
\begin{equation}
z^{*} = x - \mathrm{i}y
\end{equation}

Selain bentuk Cartesian, bilangan kompleks $z$ juga dapat dituliskan
dalam bentuk polar
\begin{align*}
z & = r\cos\theta + \mathrm{i}r\sin\theta \\
  & = r\left( \cos\theta + \mathrm{i}\sin \theta \right)
\end{align*}
dengan magnitudo atau nilai absolut atau modulus dari $z$ adalah
\begin{equation}
\left|z\right| = r = \sqrt{x^{2} + y^{2}}
\end{equation}
dan argumen atau fasa dari $z$:
\begin{equation}
\angle z = \arg(z) = \theta = \tan^{-1}\left(\frac{y}{x}\right)
\end{equation}


\section{Operasi aritmatika bilangan kompleks}

Penjumlahan:

\[
z_{1}=x_{1}+\mathrm{i}y_{1}=r_{1}\left(\cos\theta_{1}+\mathrm{i}\sin\theta_{1}\right)
\]

\[
z_{2} = x_{2} + \mathrm{\mathrm{i}}y_{2}=r_{2}\left(\cos\theta_{2}+\mathrm{i}\sin\theta_{2}\right)
\]

\[
z_{1}+z_{2}=(x_{1}+x_{2})+\mathrm{i}\left(y_{1}+y_{2}\right)
\]

Pengurangan:
\[
z_{1}+(-z_{2})=(x_{1}-x_{2})+\mathrm{i}\left(y_{1}-y_{2}\right)
\]
\[
z_{1}-z_{2}=(x_{1}-x_{2})+\mathrm{i}\left(y_{1}-y_{2}\right)
\]

Perkalian:
\begin{align*}
z_{1}z_{2} & =\left(x_{1}+\mathrm{i}y_{1}\right)\left(x_{2}+\mathrm{\mathrm{i}}y_{2}\right)\\
 & =x_{1}x_{2}+x_{1}\mathrm{i}y_{2}+\mathrm{i}y_{1}x_{2}+(\mathrm{i}y_{1})(\mathrm{i}y_{2})\\
 & =x_{1}x_{2}-y_{1}y_{2}+\mathrm{i}\left(x_{1}y_{2}+y_{1}x_{2}\right)
\end{align*}

\begin{align*}
z_{1}z_{2} & =\left(r_{1}\left(\cos\theta_{1}+\mathrm{i}\sin\theta_{1}\right)\right)\left(r_{2}\left(\cos\theta_{2}+\mathrm{i}\sin\theta_{2}\right)\right)\\
 & =r_{1}r_{2}\left(\cos\theta_{1}\cos\theta_{2}-\sin\theta_{1}\sin\theta_{2}+\mathrm{i}\left[\sin\theta_{1}\cos\theta_{2}+\cos\theta_{1}\sin\theta_{2}\right]\right)\\
 & =r_{1}r_{2}\left[\cos\left(\theta_{1}+\theta_{2}\right)+\mathrm{i}\sin\left(\theta_{1}+\theta_{2}\right)\right]
\end{align*}

\[
\frac{1}{z_{1}}=\frac{1}{x_{1}+\mathrm{i}y_{1}}\frac{x_{1}-\mathrm{i}y_{1}}{x_{1}-\mathrm{i}y_{1}}=\frac{x_{1}-\mathrm{i}y_{1}}{x_{1}^{2}+y_{1}^{2}}
\]

\[
\frac{z_{2}}{z_{1}}=...
\]

Pemangkatan

\[
z_{1}z_{2}z_{3}=r_{1}r_{2}r_{3}\left[\cos\left(\theta_{1}+\theta_{2}+\theta_{3}\right)+\mathrm{i}\sin\left(\theta_{1}+\theta_{2}+\theta_{3}\right)\right]
\]

\[
z^{3}=r^{3}\left[\cos3\theta+\mathrm{i}\sin3\theta\right]
\]

\[
z^{n}=r^{n}\left[\cos n\theta+\mathrm{i}\sin n\theta\right]
\]

$n=0,1,2,\ldots$

Tinjau kembali bentuk polar dari bilangan kompleks

\[
z=r\cos\theta+\mathrm{i}r\sin\theta
\]
yang juga dapat dituliskan

\[
z=r\cos\left(\theta+2\pi k\right)+\mathrm{i}r\sin\left(\theta+2\pi k\right)
\]
karena $\cos(\theta+2\pi)=\cos(\theta)$ dan $\sin(\theta+2\pi k)=\sin(\theta)$
untuk suatu bilangan bulat positif $k$. Oleh karena itu juga dapat
dituliskan
\[
z^{n}=r^{n}\left[\cos n\left(\theta+2\pi k\right)+\mathrm{i}\sin n\left(\theta+2\pi k\right)\right]
\]
Dengan mengubah $n\rightarrow1/n$ diperoleh:

\[
z^{1/n}=r^{1/n}\left[\cos\left(\frac{\theta+2\pi k}{n}\right)+\mathrm{i}\sin\left(\frac{\theta+2\pi k}{n}\right)\right]
\]

Formula ini dapat digunakan untuk mencari akar ke-$n$ dari suatu
bilangan kompleks $z$, misalnya solusi dari persamaan
\[
z^{3}=1
\]
adalah $1^{1/3}.$ Dengan nilai $k=0,1,2,\ldots,n-1$ terdapat $n$
akar berbeda pada rentang sudut $0\leq\theta_{k}<2\pi$, dengan $\theta_{k}\equiv\theta/n+2\pi k/n$.

Kasus untuk $z=1$, $n$-th root of unity
\[
\omega_{k}=\cos\frac{2\pi k}{n}+\mathrm{i}\sin\frac{2\pi k}{n}
\]

Contoh soal

Hitung tiga nilai dari:
\[
1^{1/3}
\]

Ubah bilangan $z=1$ menjadi bentuk kompleks: $z=r(\cos\theta+\mathrm{i}\sin\theta)$,
diperoleh $r=1$ dan $\theta=0$. Dari soal $n=3$, dan $k=0,1,2$.

Akar pertama: $n=3,k=0$
\begin{align*}
\omega_{0} & =\cos\frac{2\pi(0)}{3}+\mathrm{i}\sin\frac{2\pi(0)}{3}=1
\end{align*}

Akar kedua: $n=3,k=1$
\[
\omega_{1}=\cos\frac{2\pi(1)}{3}+\mathrm{i}\sin\frac{2\pi(1)}{3}=-\frac{1}{2}+\mathrm{i}\frac{\sqrt{3}}{2}
\]

Akar ketiga: $n=3,k=2$
\[
\omega_{2}=\cos\frac{2\pi(2)}{3}+\mathrm{i}\sin\frac{2\pi(2)}{3}=-\frac{1}{2}-\mathrm{i}\frac{\sqrt{3}}{2}
\]

Verifikasi bahwa: $\omega_{0}^{3}=1$, $\omega_{1}^{3}=1$, dan $\omega_{2}^{3}=1$






\end{document}
