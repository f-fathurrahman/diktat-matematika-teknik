\documentclass[a4paper,10pt,bahasa]{extarticle} % screen setting
\usepackage[a4paper]{geometry}

%\documentclass[b5paper,11pt,bahasa]{article} % screen setting
%\usepackage[b5paper]{geometry}

%\geometry{verbose,tmargin=1.5cm,bmargin=1.5cm,lmargin=1.5cm,rmargin=1.5cm}

\geometry{verbose,tmargin=2.0cm,bmargin=2.0cm,lmargin=2.0cm,rmargin=2.0cm}

\setlength{\parskip}{\smallskipamount}
\setlength{\parindent}{0pt}

%\usepackage{cmbright}
%\renewcommand{\familydefault}{\sfdefault}

\usepackage{amsmath}
\usepackage{amssymb}

%\usepackage[libertine]{newtxmath}
%\usepackage[no-math]{fontspec}
%\setmainfont{Linux Libertine O}

\usepackage{fontspec}
\usepackage{lmodern}
\setmonofont{JuliaMono-Regular}


\usepackage{hyperref}
\usepackage{url}
\usepackage{xcolor}
\usepackage{enumitem}
\usepackage{mhchem}
\usepackage{graphicx}
\usepackage{float}

\usepackage{minted}

\newminted{julia}{breaklines,fontsize=\footnotesize,linenos}
\newminted{python}{breaklines,fontsize=\footnotesize,linenos}

\newminted{pycon}{breaklines,fontsize=\footnotesize}

\newminted{bash}{breaklines,fontsize=\footnotesize}
\newminted{text}{breaklines,fontsize=\footnotesize}

\newcommand{\txtinline}[1]{\mintinline[breaklines,fontsize=\footnotesize]{text}{#1}}
\newcommand{\jlinline}[1]{\mintinline[breaklines,fontsize=\footnotesize]{julia}{#1}}
\newcommand{\pyinline}[1]{\mintinline[breaklines,fontsize=\footnotesize]{python}{#1}}

\newmintedfile[juliafile]{julia}{breaklines,fontsize=\footnotesize}
\newmintedfile[pythonfile]{python}{breaklines,fontsize=\footnotesize}
\newmintedfile[fortranfile]{fortran}{breaklines,fontsize=\footnotesize}

\usepackage{mdframed}
\usepackage{setspace}
\onehalfspacing

\usepackage{babel}
\usepackage{appendix}

\newcommand{\highlighteq}[1]{\colorbox{blue!25}{$\displaystyle#1$}}
\newcommand{\highlight}[1]{\colorbox{red!25}{#1}}

\newtheorem{theorem}{Teorema}
\definecolor{thmbgcol}{rgb}{0.9,0.9,0.9}
\BeforeBeginEnvironment{theorem}{
    \begin{mdframed}[backgroundcolor=thmbgcol,%
        topline=false,bottomline=false,%
        leftline=false,rightline=false]
}
\AfterEndEnvironment{theorem}{\end{mdframed}}



\newcounter{soal}%[section]
\newenvironment{soal}[1][]{\refstepcounter{soal}\par\medskip
   \noindent \textbf{Soal~\thesoal. #1} \sffamily}{\medskip}


\definecolor{mintedbg}{rgb}{0.9,0.9,0.9}
\BeforeBeginEnvironment{minted}{
    \begin{mdframed}[backgroundcolor=mintedbg,%
        topline=false,bottomline=false,%
        leftline=false,rightline=false]
}
\AfterEndEnvironment{minted}{\end{mdframed}}


\BeforeBeginEnvironment{soal}{
    \begin{mdframed}[%
        topline=true,bottomline=false,%
        leftline=true,rightline=false]
}
\AfterEndEnvironment{soal}{\end{mdframed}}



\newcounter{contoh}%[section]
\newenvironment{contoh}[1][]{\refstepcounter{soal}\par\medskip
   \noindent \textbf{Contoh~\thesoal. #1} \rmfamily}{\medskip}

\BeforeBeginEnvironment{contoh}{
    \begin{mdframed}[%
        topline=true,bottomline=false,%
        leftline=true,rightline=false]
}
\AfterEndEnvironment{contoh}{\end{mdframed}}


\renewcommand{\imath}{\mathrm{i}}

% -------------------------
\begin{document}

\title{%
{\small TF2201 Matematika Rekayasa II}\\
Variabel Kompleks: Fungsi Analitik
}
\author{Fadjar Fathurrahman\\
Teknik Fisika\\
Institut Teknologi Bandung}
\date{2023}
\maketitle


% Preview body

\section{Fungsi analitik dan Persamaan Cauchy-Riemann}


Definisi turunan fungsi dari suatu variabel kompleks $z$:
\begin{equation}
f'(z) = \lim_{\Delta z\rightarrow0}\frac{f(z+\Delta z)-f(z)}{\Delta z}    
\end{equation}
%
Perlu diperhatikan bahwa pada proses pengambilan limit, ada banyak
cara atau jalur untuk pengambilan $\Delta z$. Hal ini berbeda dalam
kasus variabel real (yang berada pada garis), karena hanya ada dua
pilihan, yaitu dari arah kiri atau arah kanan. Hal ini tidak berlaku
untuk kasus variabel kompleks yang berada pada bidang. Misalnya bisa
saja $\Delta z$ diambil dari arah sumbu $x$ positif, sumbu $x$
negatif, sumbu $y$ positif, sumbu $y$ negatif, dari arah sudut $45^{\circ}$,
dan seterusnya.

Contoh, tinjau fungsi
\begin{equation}
f(z)=z^{*}=x-\imath y
\end{equation}
Misalnya dengan menggunakan $\Delta z=\Delta x+\imath\Delta y$ diperoleh
\begin{align*}
\frac{f(z+\Delta z)-f(z)}{\Delta z} & =\frac{\left[(x+\Delta x)-\imath(y+\Delta y)\right]-(x-\imath y)}{\Delta x+\imath\Delta y}\\
 & =\frac{\Delta x-\imath\Delta y}{\Delta x+\imath\Delta y}
\end{align*}
Sekarang kita perlu menentukan bagaimana limit $\Delta z$ diambil.
Pertama misalkan $\Delta y=0$ dan $\Delta x\rightarrow0$, (limit
diambil pada arah sumbu $x$):
\begin{equation*}
\frac{\Delta x-\imath\Delta y}{\Delta x+\imath\Delta y}\rightarrow\frac{\Delta x-\imath(0)}{\Delta x+\imath(0)}=1
\end{equation*}

Kemudian misalkan $\Delta x=0$ dan $\Delta y\rightarrow0$:
\begin{equation*}
\frac{\Delta x-\imath\Delta y}{\Delta x+\imath\Delta y}\rightarrow\frac{0-\imath(\Delta y)}{0+\imath(\Delta y)}=-1
\end{equation*}

Nilai kedua limit tersebut berbeda untuk dua jalur yang berbeda. Karena
nilai limit ini tidak unik maka turunan dari $f(z)$ tidak ada.

Penentuan apakah $f'(z)$ ada melalui proses pengambilan limit, seperti
yang baru saja kita lakukan, adalah sulit dan tidak praktikal. Kita
memerlukan alternatif untuk menentukan apakah $f'(z)$ ada atau tidak.

Asumsikan bahwa $f'(z)$ ada. Bagian real dan imajiner dari $f(z)$
dinotasikan sebagai $u(x,y)$ dan $v(x,y)$, yaitu:
\begin{equation*}
f(z)=u(x,y)+\imath v(x,y)
\end{equation*}

Misalkan sekarang kita ingin mendekati $\Delta z\rightarrow0$ dari
arah sejajar dengan sumbu $x$, atau $\Delta z=\Delta x$ atau $\Delta y=0$:
\begin{align*}
f'(z) & =\lim_{\Delta x\rightarrow0}\frac{u(x+\Delta x,y) + \imath v(x+\Delta x,y)-u(x,y)-\imath v(x,y)}{\Delta x}\\
 & =\lim_{\Delta x\rightarrow0}\left[\frac{u(x+\Delta x,y)-u(x,y)}{\Delta x}+\imath\frac{v(x+\Delta x,y)-v(x,y)}{\Delta x}\right]\\
 & =\frac{\partial u}{\partial x}+\imath\frac{\partial v}{\partial x}
\end{align*}

Sekarang kita ingin mendekati $\Delta z\rightarrow0$ dari arah sejajar
dengan sumbu $y$, atau $\Delta z=\imath\Delta y$ atau $\Delta x=0$:

\begin{align*}
f'(z) & =\lim_{\Delta y\rightarrow0}\frac{u(x,y+\Delta y)+\imath v(x,y+\Delta y)-u(x,y)-\imath v(x,y)}{\imath\Delta y}\\
 & =\lim_{\Delta y\rightarrow0}\left[\frac{u(x,y+\Delta y)-u(x,y)}{\imath\Delta y}+\frac{v(x,y+\Delta y)-v(x,y)}{\Delta y}\right]\\
 & =-\imath\frac{\partial u}{\partial y}+\frac{\partial v}{\partial y}
\end{align*}

Agar $f'(z)$ memiliki turunan, maka dua kuantitas ini harus bernilai sama:
\begin{equation*}
\frac{\partial u}{\partial x}+\imath\frac{\partial v}{\partial x} = 
-\imath\frac{\partial u}{\partial y}+\frac{\partial v}{\partial y}
\end{equation*}


Dengan menyamakan bagian real dan bagian imajiner kita memperoleh
Persamaan Cauchy-Riemann:
\begin{align}
\frac{\partial u}{\partial x} & = \frac{\partial v}{\partial y} \\
\frac{\partial u}{\partial y} & = -\frac{\partial v}{\partial x}
\end{align}

Jika Persamaan Cauchy-Riemann dipenuhi pada suatu titik $z = z_{0}$,
turunan parsial pertama dari $u$ dan $v$ akan kontinu pada $z_{0}$
sehingga turunan $f'(z_{0})$ ada.

Jika $f'(z)$ ada pada $z=z_{0}$ dan pada setiap titik pada lingkungan
(\emph{neighborhood}) dari $z_{0}$, maka fungsi $f(z)$ dikatakan
bersifat \textbf{analitik} pada $z_{0}$.

Definisi dari sifat analitik dari suatu fungsi memberikan batasan
kepada himpunan di mana fungsi $f(z)$ bersifat analitik. Misalnya,
jika $f(z)$ analitik untuk semua $z$, dengan $z<1$ dan juga pada
$z=\imath$, maka $f(z)$ setidaknya analitik pada domain yang diberikan
pada Gambar XXX.

Jika $f(z)$ tidak analitik pada $z_{0}$, $f(z)$ dikatakan singular pada $z_{0}$.

Dalam banyak aplikasi, $z_{0}$ adalah titik singular yang terisolasi,
artinya di suatu lingkungan di sekitar $z_{0}$, $f(z)$ bersifat
analitik untuk $z\neq z_{0}$ dan singular hanya pada $z_{0}$. Titik
singular yang paling umum ditemukan dari suatu fungsi yang analitik
di titik lainnya (selain titik singular) adalah akar (atau zero) dari
penyebut suatu fungsi yang melibatkan rasio. Misalnya, $f(z)$ berikut
memiliki titik singular terisolasi pada $z_{0}=0$:
\begin{equation*}
\frac{1}{e^{z}-1},\frac{1}{z(z+1)},\frac{1}{z\sin(z)},\tan(z)
\end{equation*}

Fungsi rasional:
\begin{equation*}
Q(z)=\frac{a_{n}z^{n}+a_{n-1}z^{n-1}+\cdots+a_{1}z+a_{0}}{b_{m}z^{m}+b_{m-1}z^{m-1}+\cdots+b_{1}z+b_{0}}    
\end{equation*}
memiliki singularitas terisolasi pada setiap akar (atau \textit{zero}) dari
polinomial penyebut.

Untuk fungsi analitik, berlaku aturan kalkulus diferensial yang biasa
digunakan untuk variabel real.
\begin{align*}
\frac{\mathrm{d}}{\mathrm{d}z}\left[f(z)\pm g(z)\right] & = f'(z)+g'(z) \\
\frac{\mathrm{d}}{\mathrm{d}z}[kf(z)] & = kf'(z) \\
\frac{\mathrm{d}}{\mathrm{d}z}\left[f(z)g(z)\right] & = f'(z)g(z)+f(z)g'(z) \\
\frac{\mathrm{d}}{\mathrm{d}z}\left[\frac{f(z)}{g(z)}\right] & = \frac{f'(z)g(z)-f(z)g'(z)}{\left[g(z)\right]^{2}} \\
\frac{\mathrm{d}}{\mathrm{d}z}\left[f(g(z))\right] & = \frac{\mathrm{d}f}{\mathrm{d}g}\frac{\mathrm{d}g}{\mathrm{d}z}
\end{align*}

\rule[0.5ex]{1\columnwidth}{1pt}

{\color{blue}
Tentukan apakah fungsi $f(z)=zz^{*}$ bersifat analitik
}

Tulis dalam bentuk
\begin{equation*}
f(z) = zz^{*} = (x + \imath y)(x - \imath y) = x^{2} + y^{2}
\end{equation*}
sehinga $u(x,y)=x^{2}+y^{2}$ dan $v(x,y)=0$.

Turunan-turunan parsial dari $u(x,y)$ adalah 
\begin{align*}
\frac{\partial u}{\partial x} & = 2x\\
\frac{\partial u}{\partial y} & = 2y
\end{align*}
dan turunan-turunan parsial dari $v(x,y)$ adalah:
\begin{align*}
\frac{\partial v}{\partial x} & =0\\
\frac{\partial v}{\partial y} & =0
\end{align*}
Persamaan Cauchy-Riemann pertama:
\begin{align*}
\frac{\partial u}{\partial x} & =\frac{\partial v}{\partial y} \\
2x & = 0
\end{align*}
%
dan Persamaan Cauchy-Riemann kedua:
\begin{align*}
\frac{\partial u}{\partial y} & =-\frac{\partial v}{\partial x} \\
2y & =0
\end{align*}
Sehingga disimpulkan bahwa $x$ dan $y$ harus bernilai 0 agar Persamaan
Cauchy-Riemann dapat dipenuhi. Hal ini dipenuhi pada titik pusat $(0,0)$
tetapi tidak pada lingkungan disekitar titik $(0,0)$, sehingga fungsi
$f(z)=zz^{*}$ tidak analitik di manapun pada bidang kompleks.

\rule[0.5ex]{1\columnwidth}{1pt}

{\color{blue}
Tentukan apakah fungsi $f(z)=z^{2}$ bersifat analitik atau tidak.
}

Tulis $f(z)$ dalam bentuk $u(x,y) + \imath v(x,y)$ 
\begin{equation*}
f(z) = z^{2} = (x+\imath y)(x+\imath y) = x^{2} - y^{2} + \imath 2xy
\end{equation*}
sehingga
\begin{align*}
u(x,y) & = x^{2} - y^{2} \\
v(x,y) & = 2xy
\end{align*}


Persamaan Cauchy-Riemann memberikan:
\begin{align*}
\frac{\partial u}{\partial x} & =\frac{\partial v}{\partial y} \\
2x & = 2x
\end{align*}
dan
\begin{align*}
\frac{\partial u}{\partial y} & = -\frac{\partial v}{\partial x} \\
-2y & =-2y
\end{align*}

Kedua persamaan ini dipenuhi pada semua titik pada pada bidang kompleks.
Oleh karena itu $z^{2}$ bersifat analitik di titik manapun pada bidang
kompleks.

\rule[0.5ex]{1\columnwidth}{1pt}

{\color{blue}
Tentukan daerah di mana $f(z)=e^{z}$ bersifat analitik dan hitung
$f'(z)$
}

\begin{equation*}
e^{z}=e^{x+\imath y}=e^{x}e^{\imath y}=e^{x}\cos(y)+\imath e^{x}\sin(y)    
\end{equation*}
Maka:
\begin{align*}
u(x,y) & = e^{x}\cos(y)\\
v(x,y) & = e^{x}\sin(y)
\end{align*}
Dengan menghitung turunan-turunan parsial:
\begin{align*}
\frac{\partial u}{\partial x} &= e^{x}\cos(y) \\
\frac{\partial v}{\partial y} &= e^{x}\cos(y)
\end{align*}
dan
\begin{align*}
\frac{\partial u}{\partial y} & = -e^{x}\sin(y) \\
\frac{\partial v}{\partial x} & = e^{x}\sin(y)
\end{align*}
terlihat bahwa Persamaan Cauchy-Riemann terpenuhi untuk seluruh $z$.

Turunan dari $f(z)$ dapat dicari mengunakan
\begin{align*}
\frac{\mathrm{d}}{\mathrm{d}z}\left[e^{z}\right] & =\frac{\partial u}{\partial x}+\imath\frac{\partial v}{\partial x}\\
 & =e^{x}\cos(y)+\imath e^{x}\sin(y)\\
 & =e^{z}
\end{align*}

\rule[0.5ex]{1\columnwidth}{1pt}

{\color{blue}
Tentukan daerah di mana $f(z)=\ln(z)$ bersifat analitik dan hitung $f'(z)$
}

Dengan
\begin{align*}
\ln(z) & = \ln(r) + \imath\theta\\
 & = \ln\left(\sqrt{x^{2}+y^{2}}\right)+\imath\tan^{-1}\left(\frac{y}{x}\right)
\end{align*}
diperoleh
\begin{align*}
u(x,y) & =\ln\left(\sqrt{x^{2}+y^{2}}\right)=\frac{1}{2}\ln\left(x^{2}+y^{2}\right)\\
v(x,y) & =\tan^{-1}\left(\frac{y}{x}\right)
\end{align*}

Dengan menghitung turunan-turunan parsial:
\begin{align*}
\frac{\partial u}{\partial x} & = \frac{1}{2}\frac{2x}{x^{2}+y^{2}}=\frac{x}{x^{2}+y^{2}} \\
\frac{\partial v}{\partial y} & = \frac{1/x}{1+\left(y/x\right)^{2}}=\frac{x}{x^{2}+y^{2}}
\end{align*}
dan
\begin{align*}
\frac{\partial u}{\partial y} & =\frac{1}{2}\frac{2y}{x^{2}+y^{2}}=\frac{y}{x^{2}+y^{2}}\\
\frac{\partial v}{\partial x} & =-\frac{-y/x^{2}}{1+\left(y/x\right)^{2}}=-\frac{y}{x^{2}+y^{2}}
\end{align*}
%
Dapat dilihat bahwa Persamaan Cauchy-Riemann terpenuhi asalkan $x^{2}+y^{2}\neq0$
dan $\theta$ terdefinisi secara unik (menggunakan nilai pokok).

Turunan fungsi $f(z)$ adalah:
\begin{align*}
\frac{\mathrm{d}}{\mathrm{d}z}\left[\ln(z)\right] & =\frac{\partial u}{\partial x}+\imath\frac{\partial v}{\partial y}\\
 & =\frac{x}{x^{2}+y^{2}}-\imath\frac{y}{x^{2}+y^{2}}\\
 & =\frac{1}{z}
\end{align*}
yang valid dengan syarat $z\neq0$ dan $\ln(z)$ kontinu.

\rule[0.5ex]{1\columnwidth}{1pt}

{\color{blue}
Tentukan daerah di mana $f(z) = \sin(z)$ bersifat analitik dan hitung $f'(z)$
}


\section{Persamaan Laplace dan Fungsi harmonik}

Tinjau kembali Persamaan Cauchy-Riemann:
\begin{align*}
\frac{\partial u}{\partial x} & = \frac{\partial v}{\partial y} \\
\frac{\partial u}{\partial y} & = -\frac{\partial v}{\partial x}
\end{align*}

Operasikan $\dfrac{\partial}{\partial x}$ pada persamaan pertama
dan $\dfrac{\partial}{\partial y}$ pada persamaan kedua, diperoleh:
\begin{align*}
\frac{\partial^{2}u}{\partial x^{2}} &= \frac{\partial^{2}v}{\partial x\partial y} \\
\frac{\partial^{2}u}{\partial y^{2}} &= -\frac{\partial^{2}v}{\partial x\partial y}
\end{align*}

Jumlahkan kedua persamaan tersebut, diperoleh:
\begin{equation*}
\frac{\partial^{2}u}{\partial x^{2}}+\frac{\partial^{2}u}{\partial y^{2}}=0
\end{equation*}
atau:
\begin{equation}
\nabla^{2} u = 0    
\end{equation}

Sebagai alternatif, mulai dari Persamaan Cauchy-Riemann, kita juga
dapat melakukan operasi $\dfrac{\partial}{\partial x}$ pada persamaan
kedua dan $\dfrac{\partial}{\partial y}$ pada persamaan pertama,
sehingga diperoleh:
\begin{align*}
\frac{\partial^{2}u}{\partial x\partial y} &= \frac{\partial^{2}v}{\partial y^{2}} \\
\frac{\partial^{2}u}{\partial x\partial y} &= -\frac{\partial^{2}v}{\partial x^{2}}
\end{align*}
%
Kurangi persamaan pertama dengan persamaan kedua, diperoleh:
\begin{equation*}
\frac{\partial^{2}v}{\partial x^{2}}+\frac{\partial^{2}v}{\partial y^{2}}=0    
\end{equation*}
atau:
\begin{equation}
\nabla^{2}v=0    
\end{equation}

Persamaan diferensial parsial $\nabla^{2}u=0$ dan $\nabla^{2}v=0$
juga dikenal sebagai Persamaan Laplace. Hasil di atas menyatakan bahwa
bagian real dan bagian imajiner dari suatu fungsi analitik akan memenuhi
persamaan Laplace. Fungsi yang memenuhi Persamaan Laplace disebut
juga sebagai \textbf{fungsi harmonik}. Fungsi $u$ dan $v$ dari suatu
fungsi analitik dikenal juga sebagai fungsi harmonik konjugat. Jika
satu fungsi konjugat harmonik diketahui, maka fungsi konjugat harmonik
yang lain juga dapat ditemukan dengan menggunakan Persamaan Cauchy-Riemann.





\end{document}
