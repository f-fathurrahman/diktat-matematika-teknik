\begin{frame}{Sistem}

Sistem digunakan untuk memproses sinyal sehingga kita dapat memodifikasi atau
mengekstraksi informasi tambahan dari sinyal.

Suatu sistem dapat terdiri dari \textbf{komponen fisik} (realisasi
perangkat keras)
atau suatu \textbf{algoritma} yang akan menghitung sinyal output dari
sinyal input (realisasi perangkat lunak).

\end{frame}



\begin{frame}{Representasi sistem}

{\centering
\includegraphics[height=0.25\textheight]{images_priv/Lathi_Fig_1_25.png}
\par}

\end{frame}


\begin{frame}

Studi dari dari suatu sistem melibatkan tiga langkah:
\begin{itemize}
\item pemodelan matematis
\item analisis
\item desain
\end{itemize}

\end{frame}


\begin{frame}{Contoh sistem sederhana}

\fontsize{9}{10}\selectfont

\begin{columns}
  %
  \begin{column}{0.5\textwidth}
  Tinjau rangkaian listrik sederhana berikut:
  
  {\centering
  \includegraphics[width=0.8\textwidth]{images_priv/Lathi_Fig_1_26.png}
  \par}

  Input: arus $x(t)$, output: tengangan $y(t)$
  
  Tegangan dapat dihitung dari:
  $$
  y(t) = R x(t) + \frac{1}{C} \int_{-\infty}^{t} x(\tau)\,\mathrm{d}\tau
  $$
  Limit pada integral ruas persamaan kanan diambil dari $-\infty$ sampai $t$ karena integral ini
  menyatakan muatan kapasitor akibat arus $x(t)$ yang mengalir pada kapasitor.
  \end{column}
  %
  \begin{column}{0.5\textwidth}
  Pers. ini dapat dituliskan menjadi:
  $$
  y(t) = R x(t) + \frac{1}{C} \int_{-\infty}^{0} x(\tau)\,\mathrm{d}\tau + 
    \frac{1}{C} \int_{0}^{t} x(\tau)\,\mathrm{d}\tau
  $$
  Suku di tengah pada ruas kanan adalah $v_{C}(0)$, yaitu tegangan kapasitor pada $t=0$.
  Sehingga, untuk $t \geq 0$:
  $$
  y(t) = v_{C}(0) + R x(t) + \frac{1}{C} \int_{0}^{t} x(\tau)\,\mathrm{d}\tau
  $$
  Untuk sembarang waktu awal $t_0$:
  $$
  y(t) = v_{C}(t_{0}) + R x(t) + \frac{1}{C} \int_{t_{0}}^{t} x(\tau)\,\mathrm{d}\tau
  $$
  Tegangan output $y(t)$ pada waktu $t$ dapat dihitung jika kita mengetahui
  \textit{syarat awal} $v_{C}(t_0)$.
  \end{column}

\end{columns}

\end{frame}



\begin{frame}{Klasifikasi sistem}
\begin{itemize}
\item linear dan nonlinear
\item parameter konstan dan parameter bervariasi terhadap waktu
\item instant (tanpa memori) dan dinamik (dengan memori)
\item kausal dan nonkausal
\item waktu-kontinu dan waktu-diskrit
\item analog dan digital
\item invertibel dan noninvertibel
\item stabil dan nonstabil
\item deterministik dan probabilistik
\end{itemize}
\end{frame}


\begin{frame}{Konsep linearitas}

Sistem linear memiliki sifat aditif: jika beberapa input
diberikan ke sistem, maka output total akibat keseluruhan input tersebut adalah
jumlah total dari output masing-masing input independen.
Misal, pada sistem yang sama input $x_1$ memberikan output $y_1$ dan 
input $x_2$ memberikan output $y_2$, maka pada sistem linear:
input $x_1 + x_2$ akan memberikan output $y_1 + y_2$.

Sistem linear juga memenuhi sifat homogen atau sifat skala: jika input $x$
memberikan output $y$ maka input $kx$, dengan $k$ adalah suatu skalar,
adalah $ky$.

Dua sifat tersebut biasa digabung dalam satu prinsip superposisi: jika
$$
x_{1} \rightarrow y_{1} \quad \text{dan} \quad x_{2} \rightarrow y_{2}
$$
maka pada sistem linear berlaku:
$$
k_{1} x_{1} + k_{2} x_{2} \rightarrow k_{1} y_{1} + k_{2} y_{2}
$$

\end{frame}


\begin{frame}{Respon dari suatu sistem linear}

Respon input-nol (\textit{zero-input response}, ZIR) adalah respon sistem
jika input $x(t) = 0$ untuk $t \geq 0$ dengan suatu keadaan awal tertentu.
  
Respon keadaan-nol (\textit{zero-state response}, ZSR) adalah respon sistem
untuk keadaan awal diasumsikan seluruhnya bernilai nol dengan
suatu input $x(t)$ tertentu.

Jika seluruh keadaan awal bernilai nol, maka sistem dikatakan berada pada
keadaan nol (\textit{zero state}). Output atau respon sistem akan bernilai nol
ketika input nol jika dan hanya jika sistem berada pada keadaan nol.

Output atau respon dari sistem linear untuk $t \geq 0$ dapat dinyatakan sebagai
penjumlahan dua jenis respon:
$$
\text{respon total} = \text{respon input-nol} + \text{respon keadaan-nol}
$$
Sifat sistem linear ini, yang memungkinkan separasi respon menjadi komponen yang
dihasilkan dari syarat awal dan dari input, dikenal sebagai \textbf{sifat dekomposisi}.

\end{frame}


\begin{frame}

Merujuk pada contoh rangkaian listrik $RC$ yang sebelumnya dibahas:
$$
y(t) = \underbrace{v_{C}(0)}_{\text{ZIR}} +
  \underbrace{R x(t) + \frac{1}{C} \int_{0}^{t} x(\tau)\,\mathrm{d}\tau}_{\text{ZSR}}
$$

\end{frame}


\begin{frame}{Contoh}

\fontsize{9}{10}\selectfont

Apakah sistem yang dijelaskan dengan persamaan diferensial berikut ini linear?
\begin{equation}
\frac{\mathrm{d} y(t)}{\mathrm{d}t} + 3 y(t) = x(t)
\label{eq:Lathi-1-24}
\end{equation}

Misalkan untuk pasangan input-ouput $x_1$, $y_1$ dan $x_2$, $y_2$:
\begin{equation*}
\frac{\mathrm{d} y_{1}(t)}{\mathrm{d}t} + 3 y_{1}(t) = x_{1}(t) \quad \text{dan} \quad
\frac{\mathrm{d} y_{2}(t)}{\mathrm{d}t} + 3 y_{2}(t) = x_{2}(t)
\end{equation*}
Kalikan pers. pertama dengan $k_1$ dan pers. kedua dengan $k_2$, kemudian jumlahkan
kedua pers. tersebut, diperoleh:
$$
\frac{\mathrm{d}}{\mathrm{d}t} \left[
  k_{1} y_{1}(t) + k_{2} y_{2}(t) \right] +
3 \left[ k_{1} y_{1}(t) + k_{2} y_{2}(t) \right] =
k_{1} x_{1}(t) + k_{2} x_{2}(t)
$$
yang merupakan pers. diferensial \eqref{eq:Lathi-1-24} untuk input
$k_{1} x_{1}(t) + k_{2} x_{2}(t)$ dan output $k_{1} y_{1}(t) + k_{2} y_{2}(t)$.
Oleh karena itu sistem pada pers. \eqref{eq:Lathi-1-24} adalah sistem linear.

\end{frame}


\begin{frame}

Dengan generalisasi contoh sebelumnya, dapat ditunjukkan bahwa sistem yang dijelaskan
dengan pers. diferensial berbentuk:
\begin{equation}
a_0 \frac{\mathrm{d}^{N} y(t)}{\mathrm{d}t^{N}} + a_1 \frac{\mathrm{d}^{N-1} y(t)}{\mathrm{d}t^{N-1}} +
\cdots + a_{N} y(t) = b_{N-M} \frac{\mathrm{d}^{M} x(t)}{\mathrm{d}t^{M}} + \cdots +
b_{N-1} \frac{\mathrm{d} x(t)}{\mathrm{d}t} + b_{N} x(t)
\label{eq:Lathi-1-25}
\end{equation}
adalah sistem linear. Koefisien $a_{i}$ dan $b_{i}$ pada Pers. \eqref{eq:Lathi-1-25} dapat berupa
konstanta atau fungsi dari waktu.

\end{frame}


\begin{frame}{Sistem Invarian-Waktu (\textit{time-invariant}) dan
  Varian-Waktu (\textit{time-varying})}

\fontsize{10}{9}\selectfont

Sistem yang parameternya tidak berubah terhadap waktu dikatakan sebagai sistem
invarian-waktu (parameter konstan). Untuk sistem jenis ini, jika input tertunda
$T$ detik, maka output dari sistem ini juga sama, namun juga tertunda sebesar $T$ detik.

{\centering
\includegraphics[height=0.4\textheight]{images_priv/Lathi_Fig_1_29.png}
\par}

Sistem yang dijelaskan oleh Pers. \eqref{eq:Lathi-1-25} termasuk sistem linear jika
koefisien $a_i$ dan $b_i$ adalah konstanta.

Untuk sistem varian-waktu (\textit{time-varying}), hal ini tidak berlaku.
Contoh sistem varian-waktu: $y(t) = e^{-t} x(t)$.

\end{frame}



\begin{frame}{Sistem invarian-waktu}

{\centering
\includegraphics[height=0.8\textheight]{images_priv/Lathi_Fig_1_28.png}
\par}

\end{frame}


\begin{frame}{Contoh soal: sistem invarian-waktu}

Tentukan apakah sistem yang dijelaskan dengan persamaan berikut invarian-waktu
atau tidak:
\begin{enumerate}[label=(\alph*)]
\item $y(t) = x(t) \, u(t)$ di mana $u(t)$ adalah fungsi \textit{unit step}
\item $y(t) = \dfrac{\mathrm{d}}{\mathrm{d}t} x(t)$
\end{enumerate}

\end{frame}


\begin{frame}{Contoh soal: sistem invarian-waktu}

Untuk sistem yang dijelaskan dengan Pers. $y(t) = x(t) \, u(t)$
di mana $u(t)$ adalah fungsi \textit{unit step}

Pada kasus ini, output $y(t)$ akan sama dengan input untuk $t \geq 0$ dan untuk nilai $t$ yang lain
output akan bernilai 0.

Misalkan $x_1(t) = \delta(t+1)$, sehingga diperoleh $y_{1}(t) = \delta(t+1) u(t) = 0$.

Namun untuk $x_2(t) = x_{1}(t-2) = \delta(t-1)$, diperoleh $y_{2}(t) = \delta(t-1) u(t) = \delta(t-1)$

Apakah $y_2(t)$ sama dengan $y_1(t-2)$ ?

\end{frame}


\begin{frame}{Contoh soal: sistem invarian-waktu}

Untuk sistem yang dijelaskan dengan Pers. $y(t) = \dfrac{\mathrm{d}}{\mathrm{d}t} x(t)$

Untuk input $x_{1}(t)$, diperoleh:
\begin{equation*}
y_{1}(t) = \frac{\mathrm{d}}{\mathrm{d}t} x_{1}(t)
\end{equation*}

Untuk input $x_{2}(t) = x_{1}(t - T)$, dengan $T$ adalah suatu konstanta, diperoleh:
\begin{align*}
y_{2}(t) & = \frac{\mathrm{d}}{\mathrm{d}t} x_{2}(t) \\
         & = \frac{\mathrm{d}}{\mathrm{d}t} x_{1}(t-T)
\end{align*}

Apakah $y_{2}(t) = y_{1}(t-T)$ ?

\end{frame}


\begin{frame}{Sistem seketika (\textit{instantaneous}) dan dinamik}

Output dari sistem pada waktu $t$ secara umum bergantung pada keseluruhan input
sebelumnya. Meskipun demikian, pada suatu jenis sistem khusus, output pada
waktu sembarang waktu $t$ hanya bergantung pada inputnya pada saat itu juga,
misalnya pada rangkaian listrik resistif.
Sistem ini dikenal sebagai sistem seketika (\textit{instantaneous})
atau sistem tanpa memori (\textit{memoryless}).

Sistem selain sistem seketika adalah sistem dinamik atau sistem dengan memory.
Suatu sistem dinamik yang outputnya pada saat $t$ ditentukan oleh input selama
$T$ detik sebelumnya (input sejak $t-T$) dikatakan sebagai sistem dengan
memori hingga sebesar $T$.

Rangkaian listrik yang melibatkan elemen induktif dan kapasitif secara umum
memiliki memori tak hingga, karena output dari sistem tersebut ditentukan
oleh keseluruhan waktu lampau $(-\infty,t)$.

\end{frame}


\begin{frame}{Sistem kausal dan nonkausal}

Sistem kausal (dikenal juga sebagai sistem non-antisipatif atau fisis) adalah
sistem dengan ouput pada waktu $t_0$ hanya bergantung pada nilai dari input
pada $t \leq t_0$, atau output dari sistem kausal hanya bergantung pada nilai
input sekarang dan masa lalu, dan bukan pada input pada masa depan.
  
\end{frame}


\begin{frame}{Sistem waktu-kontinu dan waktu-diskrit}

Sinyal yang didefinisikan pada rentang waktu yang kontinu adalah sinyal
waktu-kontinu, biasanya dilambangkan dengan $x(t)$, $y(t)$, dan
lainnya.

Sinyal yang didefinisikan pada rentang waktu diskrit adalah
sinyal waktu-diskrit, biasanya dilambangkan dengan $x[t]$, $y[t]$,
dan lainnya.

Sistem yang input dan outputnya adalah sinyal waktu-kontinu adalah
sistem waktu-kontinu.

Sistem yang input dan outputnya adalah sinyal waktu-diskrit adalah
sistem waktu-diskrit. Komputer digital adalah contoh dari sistem waktu-diskrit.
\end{frame}



\begin{frame}{Sistem waktu-kontinu dan waktu-diskrit}

Secara praktis, sinyal waktu-diskrit dapat muncul sebagai hasil dari
pencuplikan (\textit{sampling}) dari sinyal waktu kontinu. Misalnya jika
pencuplikan dilakukan pada waktu diskrit dengan
$t_0, t_1, t_2, \ldots$ dengan jarak seragam:
$$
t_{k+1} - t_{k} = T, \quad \text{untuk seluruh } k
$$
Pada kasus tersebut, sinyal waktu diskrit direpresentasikan sebagai
sampel dari sinyal kontinu $x(t)$, $y(t)$ yang dapat dituliskan
$x(nT)$, $y(nT)$ atau $x[n] = $, $y[n]$ dengan $n$ adalah bilangan bulat.

Sinyal diskrit juga dapat muncul dari sistem yang memang terjadi
pada waktu diskrit, misalnya pada studi populasi, saham, model pendapatan nasinal,
dan sebagainya.

\end{frame}


\begin{frame}

{\centering
\includegraphics[height=0.6\textheight]{images_priv/Lathi_Fig_1_31.png}
\par}

\end{frame}


\begin{frame}

{\centering
\includegraphics[height=0.7\textheight]{images_priv/Lathi_Fig_1_32.png}
\par}
  
\end{frame}