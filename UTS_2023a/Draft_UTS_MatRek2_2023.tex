\documentclass[11pt,english]{article}

\usepackage[a4paper]{geometry}

\geometry{verbose,tmargin=2.0cm,bmargin=2.0cm,lmargin=2.0cm,rmargin=2.0cm}

\setlength{\parskip}{\smallskipamount}
\setlength{\parindent}{0pt}

\usepackage{amsmath}
\usepackage{esint}
\usepackage{babel}

\usepackage{setspace}
\onehalfspacing

\renewcommand{\imath}{\mathrm{i}}

\begin{document}

{\centering
Institut Teknologi Bandung

Program Studi Teknik Fisika

{\LARGE UJIAN TENGAH SEMESTER}

TF2201 Matematika Rekayasa II

6 Maret 2023 (08.00-10.00)
\par}


\begin{enumerate}
%
\item Hitung atau cari nilai yang memenuhi persamaan berikut.
  \begin{enumerate}
  \item $(3+4\imath)^{1/4}$ dalam bentuk Cartesian dan polar serta sketsa
  hasilnya pada bidang kompleks
  \item $(3+4\imath)^{(1-\imath)}$
  \item $\ln(-5+12\imath)$
  \item $\sin(z)=2$ (gunakan $\sin^{-1}z=-\mathrm{i}\ln\left[\mathrm{i}z+\left(1-z^{2}\right)^{1/2}\right]$)
  \end{enumerate}
%
\item Tentukan apakah fungsi-fungsi berikut ini analitik atau tidak dengan
menggunakan Persamaan Cauchy-Riemann. Jika analitik, jelaskan mengenai
daerah analitiknya.
  \begin{enumerate}
  \item $\dfrac{z}{z+4}$
  \item $e^{z-1}$
  \item $z^{3}$
  \item $z^{*}z$ ($z*$ menyatakan kompleks konjugat dari $z$)
  \end{enumerate}
%
\item Hitung integral garis atau kontur berikut.
  \begin{enumerate}
  \item $$\int_{(0,0)}^{(0,2)}z^{*}z^{2}\ \mathrm{d}z$$
  pada kurva sepanjang
  sepanjang sumbu-$x$ ke titik (2,0) kemudian diikuti dengan busur
  lingkaran dari (2,0) ke (0,2).
  \item $$\oint_{C} \dfrac{2e^{z}}{z-2}\ \mathrm{d}z$$
  dengan $C$ adalah
  lingkaran berpusat di (1,0) dan jari-jari 2.
  \end{enumerate}
%
\item Hitung integral
\[
\oint\frac{z}{z^{2}+4z-3}\ \mathrm{d}z
\]
pada kontur berikut ini.
\begin{enumerate}
\item lingkaran berpusat pada (0,0) dan jari-jari 2.
\item lingkaran berpusat pada (0,0) dan jari-jari 4.
\item lingkaran berpusat pada (2,0) dan jari-jari 2.
\item lingkaran berpusat pada (-4,0) dan jari-jari 2.
\end{enumerate}
\end{enumerate}

\end{document}
