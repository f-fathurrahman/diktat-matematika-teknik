\input{../PREAMBLE_BEAMER}

\begin{document}

\title{Engineering Mathematics II}
\subtitle{Introduction to Complex Numbers and Their Arithmetics}
\author{Fadjar Fathurrahman} %Mariya Al Qibtiya Nasution}
\date{}

\frame{\titlepage}

\begin{frame}
\frametitle{Complex number}

A complex number $z$ can be written as:
$$
z = x + \imath y
$$
with $x,y$ are two real numbers and
$$
\imath = \sqrt{-1}
$$

\end{frame}

% Cartesian plane
% abs, arg

% Polar representation

% Basic operation

\begin{frame}[fragile]
\frametitle{Complex number in Python}

Complex number in Python can be entered like this:
\begin{pythoncode}
z1 = 3 + 4j
z2 = 3.1 + 5.2j
\end{pythoncode}
Various mathematical operators already support complex arithmetics.

Mathematical functions for complex numbers are defined in standard module
\pyinline{cmath} (instead of \pyinline{math} which only supports real numbers).
\begin{pythoncode}
import cmath
zi = cmath.sqrt(-1)
\end{pythoncode}

\end{frame}


\begin{frame}[fragile]
\frametitle{Using Sympy}
  
\begin{pythoncode}
from sympy import *
init_printing()
z = 3 + 4*I
abs(z)
arg(z)
\end{pythoncode}

\end{frame}


\end{document}

