\documentclass[10pt,aspectratio=169]{beamer}
%\documentclass[fleqn,aspectratio=169]{beamer}

\usepackage{amsmath, amssymb}
\usepackage{fancyvrb, color, graphicx, hyperref, url}

%\setbeamersize{text margin left=5pt, text margin right=5pt}

\setlength{\parskip}{\smallskipamount}
\setlength{\parindent}{0pt}

\usepackage{fontspec}
%\setmonofont{DejaVu Sans Mono}
\setmonofont{JuliaMono-Regular}

\usefonttheme[onlymath]{serif}

\usepackage{minted}
\newminted{python}{breaklines,fontsize=\scriptsize}
\newminted{julia}{breaklines,fontsize=\scriptsize}
\newminted{bash}{breaklines,fontsize=\scriptsize}
\newminted{text}{breaklines,fontsize=\scriptsize}

\newcommand{\txtinline}[1]{\mintinline[breaklines,fontsize=\footnotesize]{text}{#1}}
\newcommand{\pyinline}[1]{\mintinline[breaklines,fontsize=\footnotesize]{python}{#1}}
\newcommand{\jlinline}[1]{\mintinline[breaklines,fontsize=\footnotesize]{julia}{#1}}

\definecolor{mintedbg}{rgb}{0.95,0.95,0.95}
\usepackage{mdframed}

\BeforeBeginEnvironment{minted}{\begin{mdframed}[backgroundcolor=mintedbg,%
  rightline=false,leftline=false,topline=false,bottomline=false]}
\AfterEndEnvironment{minted}{\end{mdframed}}

% https://tex.stackexchange.com/questions/33969/changing-font-size-of-selected-slides-in-beamer

\usepackage{environ}
%
% Custom font for a frame.
%
\newcommand{\customframefont}[1]{
  \setbeamertemplate{itemize/enumerate body begin}{#1}
  \setbeamertemplate{itemize/enumerate subbody begin}{#1}
}

\NewEnviron{framefont}[1]{
  \customframefont{#1} % for itemize/enumerate
  {#1 % For the text outside itemize/enumerate
    \BODY
  }
  \customframefont{\normalsize}
}

\renewcommand{\imath}{\mathrm{i}}



\begin{document}

\title{Engineering Mathematics II}
\subtitle{Introduction to Complex Numbers and Their Arithmetics}
\author{Fadjar Fathurrahman} %Mariya Al Qibtiya Nasution}
\date{}

\frame{\titlepage}

\begin{frame}
\frametitle{Complex number}

A complex number $z$ can be written as:
$$
z = x + \imath y
$$
with $x,y$ are two real numbers and
$$
\imath = \sqrt{-1}
$$

\end{frame}

% Cartesian plane
% abs, arg

% Polar representation

% Basic operation

\begin{frame}[fragile]
\frametitle{Complex number in Python}

Complex number in Python can be entered like this:
\begin{pythoncode}
z1 = 3 + 4j
z2 = 3.1 + 5.2j
\end{pythoncode}
Various mathematical operators already support complex arithmetics.

Mathematical functions for complex numbers are defined in standard module
\pyinline{cmath} (instead of \pyinline{math} which only supports real numbers).
\begin{pythoncode}
import cmath
zi = cmath.sqrt(-1)
\end{pythoncode}

\end{frame}


\begin{frame}[fragile]
\frametitle{Using Sympy}
  
\begin{pythoncode}
from sympy import *
init_printing()
z = 3 + 4*I
abs(z)
arg(z)
\end{pythoncode}

\end{frame}


\end{document}

